\newpage % Rozdziały zaczynamy od nowej strony.
\section{Wstęp}

W ramach tego rozdziału zostanie omówiony w pierwszej kolejności cel pracy. Następnie zostanie po krótce przedstawiona definicja ironii, wraz z jej podziałem na podkategorie, a także zostanie wyjaśniona różnica między ironią, a sarkazmem. W kolejnym podrozdziale zawarty został przekrój istniejący rozwiązań zajmujących się danym zagadnieniem klasyfikacji. A w ostatnim podrozdziale umieszczono krótki opis struktury pracy. 

\subsection{Cel pracy}

Celem pracy było stworzenie modelu pozwalającego na detekcję ironii i sarkazmu w tekście. Oba te sposoby wypowiedzi są traktowane jako jedna kategoria, zatem klasyfikacja odbywa się między dwoma klasami tekstów:

\begin{itemize}
    \item tekstem ironiczny lub sarkastyczny
    \item tekstem pozbawionym zarówno ironii, jak i sarkazmu
\end{itemize}

\noindent Przy czym praca, skupia się bardziej na analizie krótkich form wypowiedzi, składających się z nie więcej niż kilku zdań. Dłuższe formy wypowiedzi o charakterze ironicznym takie jak na przykład felietony wymagają innego typu analizy i nie są uwzględniane w ramach pracy. Ponadto w ramach badań podjęta jest analiza wpływu cech morfosyntaktycznych na jakość klasyfikacji modelu. Analiza ta wynika z chęci zweryfikowaniem tezy, że dodatkowe informacje na temat roli słowa w zdaniu pozwolą na ujednoznacznienie znaczenia słowa, co przełoży się na lepszą jakość analizy tekstu w tym konkretnym zadaniu klasyfikacji.

\noindent Poprzez cechy morfosyntaktyczne rozumiane są między innymi informację o tym\cite[]{Radziszewski2012}:
\begin{itemize}
    \item Do jakiej części mowy należy słowo
    \item Czy słowo występuje w liczbie pojedynczej czy mnogiej
    \item W jakim przypadku występuje słowo
    \item W jakim czasie występuje słowo
\end{itemize}



%https://www.dbc.wroc.pl/Content/23649/PDF/radziszewski_metody_PhD.pdf 


