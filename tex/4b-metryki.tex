\newpage % Rozdziały zaczynamy od nowej strony.
\section{Metryki}

//todo: zaminieć odniesienia dl CNN na odniesienia do wartwy konwolucyjnej
// bo CNN to jest cała siec wykorzystująca wartwy konwolucyjne (Conv)

Jakość klasyfikacji można monitorować przy wykorzystaniu różnych metryk. Najbardziej popularną metryką jest dokładność (ang. accuraccy), czyli stosunek poprawnie zakwalifikowanych przez model obserwacji, do całkowitej liczby obserwacji w zbiorze. Nie jest to jednak miara idealna, nie bierze ona pod uwagę między innymi tego czy badany zbiór jest zbilansowany. W przypadku zbioru niezbilansowanego możliwe jest uzyskanie bardzo dobrej dokładności nawet przy zakwalifikowaniu wszystkich obserwacji do tylko jednej klasy. W celu wyeliminowania tego problemu wykorzystuje się inne metryki, takie jak precyzja i czułość.



Powyższe metryki wykorzystują następujące pojęcia pomocnicze:
\begin{itemize}
    \item true positive (TP) - liczba pozytywnych obserwacji zaklasyfikowanych jako pozytywne
    \item true negative (TN) - liczba negatywnych obserwacji zaklasyfikowanych jako negatywne
    \item false positive (FP) - liczba negatywnych obserwacji zaklasyfikowanych jako pozytywne
    \item false negative (FN) - liczba pozytywnych obserwacji zaklasyfikowanych jako negatywne
\end{itemize}


W oparciu o te pojęcia pomocnicze można sformułować następujące zależności:

% $$
%     dokładność = \frac{\ liczba\ poprawnie\ zakwalifikowanych\ obserwacji\ }{\ całkowita\ liczba\ obserwacji\ }
% $$

$$
    precyzja = \frac{TP}{TP + FP}
$$

$$
    czułość = \frac{TP}{TP + FN}
$$


%https://developers.google.com/machine-learning/crash-course/classification/precision-and-recall 

%https://blog.exsilio.com/all/accuracy-precision-recall-f1-score-interpretation-of-performance-measures/ 

%https://towardsdatascience.com/accuracy-precision-recall-or-f1-331fb37c5cb9 

W oparciu o wzory można zauważyć, że precyzja zawiera w sobie informację jaka część obserwacji zakwalifikowanych jako pozytywne była zakwalifikowana poprawnie. Natomiast czułość zawiera informację na temat tego jaka część pozytywnych obserwacji została zakwalifikowana poprawnie.

Precyzja i czułość są ze sobą współzależne, w przypadku, gdy wskazania jednej z metryk się poprawiają, to wskazania drugiej metryki spadają. Aby monitować optymalne wskazania obu metryk powstała metryka F1, jest ona opisana poniższym wzorem:

$$
    F1 = 2 \cdot \frac{precyzja \cdot czułość}{precyzja + czułość}
$$




% $\lim_{n \to \infty}
%     \sum_{k=1}^n \frac{1}{k^2}
%     = \frac{\pi^2}{6}$
% \\ \\
% $$
%     \lim_{n \to \infty}
%     \sum_{k=1}^n \frac{1}{k^2}
%     = \frac{\pi^2}{6}
% $$




