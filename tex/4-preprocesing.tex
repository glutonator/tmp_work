% \newpage % Rozdziały zaczynamy od nowej strony.
\subsection{Wstępna obróbka danych}
Aby zapewnić jak najlepszą jakość klasyfikacji konieczne jest obrobienie danych do postaci pozbawionej zbędnych szumów. Na potrzeby tej pracy zostały podjęte następujące kroki obróbki danych:

%MOżna napisać czyimi pracami się sugerowałem pisząc wstępną obróbkę danych

\begin{enumerate}
    \item Zastąpienie linków tagiem “<url>”.
    \item Usunięcie znaków końca linii.
    \item Zastąpienie wystąpień nazw użytkowników tagiem  “<username>”.
    \item Zamiana emotikon z formatu “:XX\_YY:” na format “<emote> XX YY” .
    \item Rozbicie hasztagów na mniejsze słowa, oznaczenie go tagiem “<hashtag>”.
    \item Zastąpienie oznaczeń odnoszących się do czasu tagiem “<time>”.
    \item Zastąpienie liczb tagiem “<number>”.
    \item Usunięcie znaków  '' | ''  oraz wielokropków.
    \item Konwersja do małych liter.
    \item Rozwinięcie form skróconych do ich pełnych form.
\end{enumerate}

\begin{table}[!h] \label{tab:tabela2} \centering
    \caption{Tabela przedstawiająca efekt wstępnej obróbki danych.}
    \begin{tabular} {| c | c | c |} \hline
        Numer operacji          & Przed konwersją                                          & Po konwersji                   \\
        dokonanej w ramach      &                                                          &                                \\
        wstępnej obróbki danych &                                                          &                                \\ \hline\hline
        1                       & www.google.pl                                            & <url>                          \\ \hline
        2                       & \textbackslash n \quad \textbackslash r \textbackslash n &                                \\ \hline
        3                       & @mike                                                    & <username>                     \\ \hline
        4                       & :face\_screaming\_in\_fear:                              & <emote> face screaming in fear \\ \hline
        5                       & \#trueFriend                                             & <hashtag> true Friend          \\ \hline
        6                       & 14:58                                                    & <time>                         \\ \hline
        7                       & 1234                                                     & <number>                       \\ \hline
        8                       & | ...                                                    & .                              \\ \hline
        9                       & Kazimierz Wielki                                         & kazimierz wielki               \\ \hline
        10                      & I'll \quad | \quad He's                                  & I will \quad | \quad He has    \\ \hline
    \end{tabular}
\end{table}

%todo:
% być może warto opisać w jaki sposób robię te operacje usuwania ?? -> że regexxy
% być może warto opisać o tych formach skróconych, że to model dodakowy wychwytuje
% można teżnapisać czemu nie wykorzystuję stremingu i lematyzacji
%


% \begin{tabular}{|l|l|}
%     \hline
%     Aaaaaaaaaa & yyy \\
%     xxxxx      & B   \\  \hline
%     D          & E   \\    \hline
% \end{tabular}


