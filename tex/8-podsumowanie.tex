% PODSUMOWANIE

W ramach pracy została dokonana kompleksowa analiza pozwalająca na zbudowanie modelu sieci neuronowej umożliwiającej rozwiązanie zadania klasyfikacji tekstów ironicznych. Zostały w niej zaprezentowane kolejne kroki, które należy wykonać by pozwolić sieci na interpretacje zdania. W pierwszej kolejności zostały omówione czynności niezbędne w ramach obróbki wstępnej danych. Następnie, przedstawiono sposoby przeniesienia zmiennych kategorycznych w postaci słów do przestrzeni wektorów liczbowych przy zachowaniu jak największej liczby informacji na temat podobieństwa poszczególnych słów. W końcowym etapie porównano kilka możliwych architektur sieci i ich skuteczność poprzez wykorzystanie różnych popularnych metryk. Ponadto zbadano wpływ cech morfosyntaktycznych na jakość klasyfikacji w problemie detekcji ironii. W oparciu o drobiazgowe wyniki nie dostrzeżono znaczącej poprawy jakości klasyfikacji dla żadnego z wykorzystanych zbiorów. 

Niestety w ramach pracy nie wszystko poszło zgodnie z założeniami początkowymi. Nie udało się na przykład dokonać detekcji ironii w ramach języka polskiego. Było to spowodowane w dużej mierze brakiem łatwo dostępnych, bezpłatnych danych mogących posłużyć do analizy. W przypadku rozwiązania tego problemu, otwiera się bardzo ciekawy kierunek dalszego rozwoju tej pracy. 

Innym elementem mogącym być dalszym kierunkiem rozwoju pracy jest opracowanie modelu pozwalającego na większą generalizację klasyfikacji dokonywanej przez model. W obecnej konfiguracji, sieć nie rodzi sobie dobrze z klasyfikacją wypowiedzi pochodzących z poza korpusu wykorzystanego w procesie treningu i testowania. Najprawdopodobniej wynika to ze zbyt małej różnorodności wypowiedzi w ramach zbioru danych. W celu rozwiązania tego problemu należałoby stworzyć korpus o bardziej różnorodnej budowie. 