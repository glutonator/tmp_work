% \newpage % Rozdziały zaczynamy od nowej strony.
% \section{Wstęp}
% \subsection{Cel pracy}

Praca dąży do zaproponowania modelu zdolnego do klasyfikacji tekstów o charakterze ironicznym. Aby to było możliwe, praca w pierwszej kolejności analizuje istniejące już podejścia do rozwiązania takiego zadania klasyfikacji. By następnie w oparciu o zdobytą wiedzę zaproponować kilka architektur sieci najlepiej nadających się do rozwiązania tego problemu. 

Praca skupia się na analizie sieci CNN i sieci opartych na warstwach LSTM oraz na doborze konfiguracji hiper parametrów pozwalającej osiągnąć jak najlepszą dokładność klasyfikacji. Aby to było możliwe w pierwszej kolejności dokonywana jest wstępna obróbka danych. W trakcie niego część elementów zbioru jest zastępowana bardziej generalnymi oznaczeniami, aby zminimalizować poziom szumu w zbiorze uczącym. Ponadto w trakcie tego kroku następuje konwersja emotikon i hasztagów do formy lepiej interpretowalnej przez modele językowe. W następnej kolejności praca skupia się na transformacji danych w formie słów do przestrzeni liczbowej, która może być interpretowana przez sieci neuronowe. W tym celu dokonana jest analiza istniejących sposobów uzyskiwania reprezentacji wektorowej słów i wybór kilku z nich, najlepiej pozwalających na oddanie cech słów w przestrzeni wektorowej. 

Praca, oprócz zaproponowania modelu pozwalającego na detekcję ironii, skupia się także na analizie wpływu oznaczeń morfosyntaktycznych na jakość klasyfikacji. Stara się odpowiedzieć na pytanie, czy taka informacja jest istotna dla sieci w procesie nauki, czy wprowadza może tylko nie istotne szumy. 


% Celem pracy było stworzenie modelu pozwalającego na detekcję ironii i sarkazmu w tekście. Przy czym praca, skupia się bardziej na analizie krótkich form wypowiedzi, składających się z nie więcej niż kilku zdań. Dłuższe formy wypowiedzi o charakterze ironicznym takie jak na przykład felietony wymagają innego typu analizy i nie są uwzględniane w ramach pracy. Ponadto w ramach badań podjęta jest analiza wpływu cech morfosyntaktycznych na jakość klasyfikacji modelu. Analiza ta wynika z chęci zweryfikowaniem tezy, że dodatkowe informacje na temat roli słowa w zdaniu pozwolą na ujednoznacznienie znaczenia słowa, co przełoży się na lepszą jakość analizy tekstu w tym konkretnym zadaniu klasyfikacji.

% Poprzez cechy morfosyntaktyczne rozumiane są między innymi informację o tym:
% \begin{itemize}
%     \item Do jakiej części mowy należy słowo
%     \item Czy słowo występuje w liczbie pojedynczej czy mnogiej
%     \item W jakim przypadku występuje słowo
%     \item W jakim czasie występuje słowo
% \end{itemize}



%https://www.dbc.wroc.pl/Content/23649/PDF/radziszewski_metody_PhD.pdf 


