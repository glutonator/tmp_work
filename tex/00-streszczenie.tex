% \newpage % Rozdziały zaczynamy od nowej strony.
% \section{Wstęp}
% \subsection{Cel pracy}

Zagadnienia z dziedziny przetwarzania tekstu naturalnego były kiedyś zdominowane przez systemy regułowe, które determistycznie rozwiązywały postawione im problemy. Aktualnie, dla tych samych problemów, coraz częściej wykorzystywane są systemu sztucznej inteligencji oparte o sieci neuronowe. Pozwalają one na wychwycenie relacji nie zawsze intuicyjnych dla człowieka, co czasem pozwala na skuteczniejsze rozwiązanie postawionego im problemu. Jednak ta cecha może prowadzić także do przeciwnych rezultatów m. in. w przypadku źle dobranego zbioru danych, dlatego przy takim podejściu bardzo ważne jest dokładne testowanie powstałego modelu. 

Praca skupia się na zagadnieniu przetwarzania tekstu naturalnego w zakresie klasyfikacji go do jednej z dwóch grup, zawierającej ironię lub sarkazm oraz niezawierającej ani ironii, ani sarkazmu. Drugim celem pracy była analiza wpływu cech morfosyntaktycznych na jakość klasyfikacji tekstów do jednej z wcześniej wymienionych grup.  

W ramach pracy wykorzystano istniejące zbiory danych, na których została dokonana operacja obróbki wstępnej oczyszczająca zbiór z nieistotnych danych. Następnie tak przetworzone dane zostały konwertowane do reprezentacji wektorowej wykorzystując trzy różne metody. W ramach tego kroku została także uwzględniona informacja na temat cech morfosyntaktycznych słow. Kolejnym krokiem było wprowadzenie tak zakodowanej informacji do sieci neuronowej. W ramach pracy zaproponowano kilka architektur bazujących na warstwach konwolucyjnych i LSTM. 

W ramach pracy udało się skonstruować modele pozwalające na rozwiązanie postawionego zadania klasyfikacji na dość dobrym poziomie. Dokładność klasyfikacji różniła się istotnie w zależności od wykorzystanego zbioru danych. Co wskazuje, że niektóre typy ironii i sarkazmu mogą być gorzej rozpoznawalne przez powstałe modele. Ponadto nie zauważono istotnej różnicy w jakości klasyfikacji między najskuteczniejszymi sieciami opartymi na warstwach konwolucyjnych i warstwach LSTM. Natomiast analiza wpływu cech morfosyntaktycznych na dokładność klasyfikacji wykazała brak istotnej poprawy jakości działania modelu. Skłania to do wniosków, że takie informacje nie są istotne w ramach detekcji ironii czy sarkazmu.  


%https://www.dbc.wroc.pl/Content/23649/PDF/radziszewski_metody_PhD.pdf 


