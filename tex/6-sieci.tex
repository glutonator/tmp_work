\newpage % Rozdziały zaczynamy od nowej strony.
\section{Szczegółowy opis wykorzystanych architektur sieci}

W tym rozdziale przedstawiona jest dokładna lista wykorzystanych w pracy architektur sieci, ich lista znajduje się w tabeli \ref{table:architektury_sieci}. Nastomiast w tableli \ref{tab:oznaczenia_warstw} został umieszczony opis skrótów zastosowanych do opisu poszczególnych wartw.

% WAŻNE: TODO / FYI The \label should always be after \caption:

%todo: dodać zastąpienie dla XXX

\begin{table}[!h]  \centering
    \caption{Opis oznaczeń wykorzsytanych w tabeli \ref{table:architektury_sieci}}
    \label{tab:oznaczenia_warstw}
    \begin{tabular} {| l | l |} \hline
        Oznaczenie    & Opis oznaczeń                                                          \\ \hline
        Conv(n,k)     & wartwa konwolucyjna, gdzie                                             \\ & \textbf{n} - liczba zasosowanych filtrów, \\ & \textbf{k} - szerokość zastosowanego filtru \\ \hline
        Flatten       & warstwa powodująca spłaszczenie sieci do postaci jednowymiarowej       \\ \hline
        MaxPooling(f) & wartwa max poolingu o oknie wielkości \textbf{f}                                \\ & oraz skoku przesunięcia także wynoszacego \textbf{f}   \\ \hline
        Dense(n)      & wartwa gęsta, o liczbie nueronów wynoszącej \textbf{n}                 \\ \hline
        LSTM(n)       & wartwa LSTM, liczbie nueronów wynoszącej \textbf{n}                            \\ \hline
        Bi-LSTM(n)    & wartwa Bi-LSTM, w której każda warstwa LSTM posiada \textbf{n} neuronów         \\ \hline
        Dropout(d)    & parametr określający jak duża część połączeń zostanie  usunięta między \\ & kolejnymi wartwami, wartość paramtru \textbf{d} określa jaka część losowych \\ & połączeń będzie usunięta w trakcie jednego kroku uczenia, \\ & zawiera się w przedziale od 0 do 1                                                             \\ \hline
    \end{tabular}
\end{table}

\begin{longtable}{| c | m{0.58\linewidth} | r | m{0.1\linewidth} |}
    \caption{Architektury sieci wykorzystane w ramach pracy}
    \label{table:architektury_sieci}        \\
    \hline
    Nazwa sieci & \multicolumn{1}{c|}{Sieć} \\ \hline\hline \endfirsthead

    \endfoot
    \hline \endlastfoot

    $cnn\_1$    & Conv(50,3)                \\
                & Flatten                   \\
                & Dense(50)                 \\
                & Dense(1)                  \\ \hline
    $cnn\_2$    & Conv(300,3)               \\
                & Flatten                   \\
                & Dense(300)                \\
                & Dense(1)                  \\ \hline
    $cnn\_3$    & Conv(300,15)              \\
                & Flatten                   \\
                & Dense(300)                \\
                & Dense(1)                  \\ \hline
    $cnn\_4$    & Conv(300,15)              \\
                & MaxPooling(2)             \\
                & Flatten                   \\
                & Dense(300)                \\
                & Dense(1)                  \\ \hline
    $cnn\_5$    & Conv(300,15)              \\
                & MaxPooling(8)             \\
                & Flatten                   \\
                & Dense(300)                \\
                & Dense(1)                  \\ \hline
    $cnn\_6$    & Conv(300,3)               \\
                & MaxPooling(2)             \\
                & Conv(300,3)               \\
                & MaxPooling(2)             \\
                & Flatten                   \\
                & Dense(300)                \\
                & Dense(1)                  \\ \hline

    $cnn\_7$    & Conv(300,3)               \\
                & MaxPooling(2)             \\
                & Conv(300,3)               \\
                & MaxPooling(2)             \\
                & Conv(300,3)               \\
                & MaxPooling(2)             \\
                & Flatten                   \\
                & Dense(300)                \\
                & Dense(1)                  \\ \hline
    $cnn\_8$    & Conv(300,5)               \\
                & Flatten                   \\
                & Dense(300)                \\
                & Dense(1)                  \\ \hline
    $cnn\_9$    & Conv(300,7)               \\
                & Flatten                   \\
                & Dense(300)                \\
                & Dense(1)                  \\ \hline
    $cnn\_10$   & Conv(300,9)               \\
                & Flatten                   \\
                & Dense(300)                \\
                & Dense(1)                  \\ \hline


    $cnn\_11$   & Conv(500,3)               \\
                & Flatten                   \\
                & Dense(500)                \\
                & Dense(1)                  \\ \hline

    %%%%%%%%%%%%%%%%%%%%%%%%%%%%%%%%%%%%%%%%%%%%%%%%%%%%%%%%%%5

    $dense\_1$  & Dense(30)                 \\
                & Flatten                   \\
                & Dense(1)                  \\ \hline

    $dense\_2$  & Dense(30)                 \\
                & Dense(30)                 \\
                & Flatten                   \\
                & Dense(1)                  \\ \hline

    $dense\_3$  & Dense(300)                \\
                & Dense(300)                \\
                & Flatten                   \\
                & Dense(1)                  \\ \hline

    $dense\_4$  & Dense(300)                \\
                & Dropout(0.2)              \\
                & Dense(300)                \\
                & Dropout(0.2)              \\
                & Flatten                   \\
                & Dense(1)                  \\ \hline
    %%%%%%%%%%%%%%%%%%%%%%%%%%%%%%%%%%%%%%%%%%%%%%%%%%%%%%%%%%
    $lstm\_1$   & LSTM(50)                  \\
                & Dense(5)                  \\
                & Dense(1)                  \\ \hline

    $lstm\_2$   & LSTM(50)                  \\
                & Dense(50)                 \\
                & Dense(1)                  \\ \hline

    $lstm\_3$   & LSTM(30)                  \\
                & Dense(30)                 \\
                & Dense(1)                  \\ \hline
    $lstm\_4$   & LSTM(30)                  \\
                & Dense(1)                  \\ \hline
    $lstm\_5$   & LSTM(30)                  \\
                & Dense(1)                  \\ \hline
    $lstm\_6$   & Bi-LSTM(50))              \\
                & Dense(1)                  \\ \hline
    $lstm\_7$   & Bi-LSTM(25))              \\
                & Dense(1)                  \\ \hline

    %%%%%%%%%%%%%%%%%%%%%%%%%%%%%%%%%%%%%%%%%%%%%%%%%%%%%%%%%%

    $lstm\_8$   & LSTM(300)                 \\
                & Dense(300)                 \\
                & Dense(1)                  \\ \hline

    $lstm\_9$   & LSTM(300)                 \\
                & LSTM(300)                 \\
                & Dense(300)                \\
                & Dense(1)                  \\ \hline

    $lstm\_10$  & LSTM(300)                 \\
                & Dropout(0.2)              \\
                & LSTM(300)                 \\
                & Dropout(0.2)              \\
                & Dense(300)                \\
                & Dense(1)                  \\ \hline
    $lstm\_11$  & LSTM(300)                 \\
                & Dropout(0.2)              \\
                & LSTM(300)                 \\
                & Dropout(0.2)              \\
                & LSTM(300)                 \\
                & Dropout(0.2)              \\
                & Dense(300)                \\
                & Dense(1)                  \\ \hline
    $lstm\_12$  & Bi-LSTM(300)              \\
                & Dropout(0.2)              \\
                & Bi-LSTM(300)              \\
                & Dropout(0.2)              \\
                & Dense(300)                \\
                & Dense(1)                  \\ \hline
    $lstm\_13$  & Bi-LSTM(300)              \\
                & Dropout(0.2)              \\
                & Bi-LSTM(300)              \\
                & Dropout(0.2)              \\
                & Dense(300)                \\
                & Dense(1)                  \\ \hline

    $lstm\_14$  & Bi-LSTM(500)              \\
                & Dropout(0.2)              \\
                & Dense(500)                \\
                & Dense(1)                  \\ \hline

    $lstm\_15$  & Bi-LSTM(300)              \\
                & Dropout(0.2)              \\
                & Bi-LSTM(300)              \\
                & Dropout(0.2)              \\
                & Dense(300)                \\
                & Dense(1)                  \\ \hline

    $lstm\_16$  & Bi-LSTM(300)              \\
                & Dropout(0.2)              \\
                & Bi-LSTM(300)              \\
                & Dropout(0.2)              \\
                & Bi-LSTM(300)              \\
                & Dropout(0.2)              \\
                & Dense(300)                \\
                & Dense(1)                  %\\ \hline
\end{longtable}



