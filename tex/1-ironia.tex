% \newpage % Rozdziały zaczynamy od nowej strony.
\subsection{Ironia}

Ironia jest to sposób wypowiadania się, oparty na zamierzonej niezgodności, najczęściej przeciwieństwie, dwóch poziomów wypowiedzi: dosłownego i ukrytego. W potocznych wypowiedziach utożsamia się ją z zawoalowaną kpiną, złośliwością, czy wyśmiewaniem.

% http://blog.flocabulary.com/definitions-and-examples-of-irony-in-literature/ 

% http://typesofirony.com/the-3-types-of-irony/ 

Do najpopularniejszych rodzajów ironii należą:
\begin{itemize}
    \item Ironia sytuacyjna - wiąże się z rozbieżnością między tym jakie są oczekiwania co do tego co powinno się wydarzyć, a tym co faktycznie się wydarzyło np. “Na konferencji na temat IT nie było dostępu do Internetu”
    \item Ironia werbalna - występuje, gdy intencja (przekaz) osoby wypowiadającej się jest inna niż wynikałoby z dosłownego rozumienia słów w wypowiedzi np. “Nie mogę się już doczekać by wreszcie przeczytać ten siedmiuset stronicowy raport”
    \item Dramatyczna ironia - jest to rodzaj ironii popularny w filmach, książkach i sztukach teatralnych, występuje, gdy widz/czytelnik posiada kluczowa informację, której nie zna bohater utworu np. “tragiczna śmierć Romea z utworu ‘Romeo i Julia’, który nie jest świadomy, że jego ukochana tak naprawdę nie umarła, a zapadła w stan tylko z pozoru przypominający śmierć”
\end{itemize}


Zjawiskiem często łączącym się z ironią jest sarkazm. Sarkazm jest formą ironii mającej na celu skrytykowanie zaistniałej sytuacji lub obrażenie danej osoby.
\begin{itemize}
    \item Przykład ironii: “Super, ktoś poplamił moją nową sukienkę”
    \item Przykład sarkazmu: “Nazywasz to coś dziełem sztuki?”
\end{itemize}



