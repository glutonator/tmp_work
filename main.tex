%%%%%%%%%%%%%%%%%%%%%%%%%%%%%%%%%%%%%%%%%%%%%%%%%%%%%%%
%% Engineer & Master Thesis, LaTeX Template          %%
%% Copyleft by Piotr Woźniak & Artur M. Brodzki      %%
%% Faculty of Electronics and Information Technology %%
%% Warsaw University of Technology, Warsaw, 2019     %%
%%%%%%%%%%%%%%%%%%%%%%%%%%%%%%%%%%%%%%%%%%%%%%%%%%%%%%%

\documentclass[
    left=2.5cm,         % Sadly, generic margin parameter
    right=2.5cm,        % doesnt't work, as it is
    top=2.5cm,          % superseded by more specific
    bottom=3cm,         % left...bottom parameters.
    bindingoffset=6mm,  % Optional binding offset.
    nohyphenation=false % You may turn off hyphenation, if don't like.
]{eiti/eiti-thesis}

\langpol % Dla języka angielskiego mamy \langeng
\graphicspath{{img/}}             % Katalog z obrazkami.
\addbibresource{bibliografia.bib} % Plik .bib z bibliografią

\begin{document}

%--------------------------------------
% Strona tytułowa
%--------------------------------------
\MasterThesis % dla pracy inżynierskiej mamy \EngineerThesis
\instytut{Instytut Automatyki i Informatyki Stosowanej}
\kierunek{Informatyka}
\specjalnosc{Systemy Informacyjno-Decyzyjne}
\title{
    Klasyfikacja tekstów ironicznych
}
\engtitle{ % Tytuł po angielsku do angielskiego streszczenia
    Classification of ironic texts.
}
\author{Filip Lewczak}
\album{259024}
\promotor{dr inż. Mariusz Kamola}
\date{\the\year}
\maketitle

%--------------------------------------
% Streszczenie po polsku
%--------------------------------------
\cleardoublepage % Zaczynamy od nieparzystej strony
\streszczenie  
% //TODO 
% \newpage % Rozdziały zaczynamy od nowej strony.
% \section{Wstęp}
% \subsection{Cel pracy}

Praca dąży do zaproponowania modelu zdolnego do klasyfikacji tekstów o charakterze ironicznym. Aby to było możliwe, praca w pierwszej kolejności analizuje istniejące już podejścia do rozwiązania takiego zadania klasyfikacji. By następnie w oparciu o zdobytą wiedzę zaproponować kilka architektur sieci najlepiej nadających się do rozwiązania tego problemu. 

Praca skupia się na analizie sieci CNN i sieci opartych na warstwach LSTM oraz na doborze konfiguracji hiper parametrów pozwalającej osiągnąć jak najlepszą dokładność klasyfikacji. Aby to było możliwe w pierwszej kolejności dokonywana jest wstępna obróbka danych. W trakcie niego część elementów zbioru jest zastępowana bardziej generalnymi oznaczeniami, aby zminimalizować poziom szumu w zbiorze uczącym. Ponadto w trakcie tego kroku następuje konwersja emotikon i hasztagów do formy lepiej interpretowalnej przez modele językowe. W następnej kolejności praca skupia się na transformacji danych w formie słów do przestrzeni liczbowej, która może być interpretowana przez sieci neuronowe. W tym celu dokonana jest analiza istniejących sposobów uzyskiwania embeddingów i wybór kilku z nich, najlepiej pozwalających na oddanie cech słów w przestrzeni wektorowej. 

Praca, oprócz zaproponowania modelu pozwalającego na detekcję ironii, skupia się także na analizie wpływu oznaczeń morfosyntaktycznych na jakość klasyfikacji. Stara się odpowiedzieć na pytanie, czy taka informacja jest istotna dla sieci w procesie nauki, czy wprowadza może tylko nie istotne szumy. 


% Celem pracy było stworzenie modelu pozwalającego na detekcję ironii i sarkazmu w tekście. Przy czym praca, skupia się bardziej na analizie krótkich form wypowiedzi, składających się z nie więcej niż kilku zdań. Dłuższe formy wypowiedzi o charakterze ironicznym takie jak na przykład felietony wymagają innego typu analizy i nie są uwzględniane w ramach pracy. Ponadto w ramach badań podjęta jest analiza wpływu cech morfosyntaktycznych na jakość klasyfikacji modelu. Analiza ta wynika z chęci zweryfikowaniem tezy, że dodatkowe informacje na temat roli słowa w zdaniu pozwolą na ujednoznacznienie znaczenia słowa, co przełoży się na lepszą jakość analizy tekstu w tym konkretnym zadaniu klasyfikacji.

% Poprzez cechy morfosyntaktyczne rozumiane są między innymi informację o tym:
% \begin{itemize}
%     \item Do jakiej części mowy należy słowo
%     \item Czy słowo występuje w liczbie pojedynczej czy mnogiej
%     \item W jakim przypadku występuje słowo
%     \item W jakim czasie występuje słowo
% \end{itemize}



%https://www.dbc.wroc.pl/Content/23649/PDF/radziszewski_metody_PhD.pdf 



%\lipsum[1-3]
\slowakluczowe ironia, NLP, analiza morfosyntaktyczna

%--------------------------------------
% Streszczenie po angielsku
%--------------------------------------
\newpage
\abstract //TODO
 
 % \kant[1-3]
\keywords XXX, XXX, XXX

%--------------------------------------
% Oświadczenie o autorstwie
%--------------------------------------
\newpage
\makeauthorship

%--------------------------------------
% Spis treści
%--------------------------------------
\cleardoublepage % Zaczynamy od nieparzystej strony
\pagestyle{plain}
\tableofcontents

%--------------------------------------
% Rozdziały
%--------------------------------------
\cleardoublepage % Zaczynamy od nieparzystej strony
\pagestyle{headings}

% Wygodnie jest trzymać każdy rozdział w osobnym pliku.
% Umożliwia to również łatwą migrację do nowej wersji szablonu:
% wystarczy podmienić swoje pliki main.tex i eiti-thesis.cls
% na nowe wersje, a cały tekst pracy pozostaje nienaruszony.
\newpage % Rozdziały zaczynamy od nowej strony.
\section{Wstęp}
\subsection{Cel pracy}

Celem pracy było stworzenie modelu pozwalającego na detekcję ironii i sarkazmu w tekście. Przy czym praca, skupia się bardziej na analizie krótkich form wypowiedzi, składających się z nie więcej niż kilku zdań. Dłuższe formy wypowiedzi o charakterze ironicznym takie jak na przykład felietony wymagają innego typu analizy i nie są uwzględniane w ramach pracy. Ponadto w ramach badań podjęta jest analiza wpływu cech morfosyntaktycznych na jakość klasyfikacji modelu. Analiza ta wynika z chęci zweryfikowaniem tezy, że dodatkowe informacje na temat roli słowa w zdaniu pozwolą na ujednoznacznienie znaczenia słowa, co przełoży się na lepszą jakość analizy tekstu w tym konkretnym zadaniu klasyfikacji.

Poprzez cechy morfosyntaktyczne rozumiane są między innymi informację o tym:
\begin{itemize}
    \item Do jakiej części mowy należy słowo
    \item Czy słowo występuje w liczbie pojedynczej czy mnogiej
    \item W jakim przypadku występuje słowo
    \item W jakim czasie występuje słowo
\end{itemize}



%https://www.dbc.wroc.pl/Content/23649/PDF/radziszewski_metody_PhD.pdf 



% \newpage % Rozdziały zaczynamy od nowej strony.
\subsection{Ironia}

Ironia jest to sposób wypowiadania się, oparty na zamierzonej niezgodności, najczęściej przeciwieństwie, dwóch poziomów wypowiedzi: dosłownego i ukrytego. W potocznych wypowiedziach utożsamia się ją z zawoalowaną kpiną, złośliwością, czy wyśmiewaniem.

% http://blog.flocabulary.com/definitions-and-examples-of-irony-in-literature/ 

% http://typesofirony.com/the-3-types-of-irony/ 

Do najpopularniejszych rodzajów ironii należą:
\begin{itemize}
    \item Ironia sytuacyjna - wiąże się z rozbieżnością między tym jakie są oczekiwania co do tego co powinno się wydarzyć, a tym co faktycznie się wydarzyło np. “Na konferencji na temat IT nie było dostępu do Internetu”
    \item Ironia werbalna - występuje, gdy intencja (przekaz) osoby wypowiadającej się jest inna niż wynikałoby z dosłownego rozumienia słów w wypowiedzi np. “Nie mogę się już doczekać by wreszcie przeczytać ten siedmiuset stronicowy raport”
    \item Dramatyczna ironia - jest to rodzaj ironii popularny w filmach, książkach i sztukach teatralnych, występuje, gdy widz/czytelnik posiada kluczowa informację, której nie zna bohater utworu np. “tragiczna śmierć Romea z utworu ‘Romeo i Julia’, który nie jest świadomy, że jego ukochana tak naprawdę nie umarła, a zapadła w stan tylko z pozoru przypominający śmierć”
\end{itemize}


Zjawiskiem często łączącym się z ironią jest sarkazm. Sarkazm jest formą ironii mającej na celu skrytykowanie zaistniałej sytuacji lub obrażenie danej osoby.
\begin{itemize}
    \item Przykład ironii: “Super, ktoś poplamił moją nową sukienkę”
    \item Przykład sarkazmu: “Nazywasz to coś dziełem sztuki?”
\end{itemize}




% \newpage % Rozdziały zaczynamy od nowej strony.
\subsection{Analiza literatury}

Zagadnienie klasyfikacji ironii jest dość popularnym problemem wśród badaczy zajmujących się analizą języka naturalnego ze względu na konieczność wychwycenia nie tylko dosłownego znaczenia danej wypowiedzi, ale także detekcję i analizy ewentualnego podtekstu.

% //analiza sentymentu = analiza wydzwięku emeocjonalnego

Pierwsze prace z tej dziedziny opierały się na analizie sentymentu poszczególnych słów. Opierały się na przeświadczeniu, że jeśli wypowiedź utrzymuje stały charakter emocjonalny w postaci pozytywnego lub negatywnego wydźwięku    nie zawiera ono w sobie ironii. Natomiast jeśli pojawiały się fragmenty o przeciwnej polaryzacji, to zdanie z dużym prawdopodobieństwem można było klasyfikować jako ironiczne. Takie metody skupiały się przede wszsytkim na kryteriach ilościowych, nie uwględniając większej istotności słów kluczowych. Z tego powodu metody tego typu były dość zawodne.

% Metody te jednak były dość zawodne ze względu na brak uwzględniania znaczenia poszczególnych słów, pozwalające na …//todo

% //Odnieśc się ->  Sarcasm as Contrast between a Positive Sentiment and Negative Situation

% //todo: opisać systemy regułowe


Inne podejście, zaprezentowane w pracy \cite{Riloff2013} opierało się na spostrzeżeniach, że w zdaniach sarkastycznych słowa o pozytywnym wydźwięku kontrastują z sytuacjami nacechowanymi negatywnie. W oparciu o tą tezę badacze zaproponowali algorytm systematycznie uczący się dwóch grup sformułowań:

\begin{itemize}
    \item słów o pozytywnym wydźwięku (oznaczony jako PS)
    \item sformułowań świadczących o sytuacjach nacechowanych negatywnie (oznaczony jako NS)
\end{itemize}


Algorytm rozpoczyna swoje działanie od tylko jednego słowa ‘love’ w zbiorze słów PS. Następnie dokonywane są kolejne kroki:
\begin{enumerate}
    \item Wyszukiwanie w zbiorze danych rekordów zawierających słowo ze zbioru PS
    \item Dla każdego znalezionego rekordu w otoczeniu słowa ze zbioru PS wyznaczany jest 1-gram,2-gram oraz 3-gram.
    \item Ze stworzonych n-gramow wybierany jest ten, który występuje najczęściej dla rekordów sarkastycznych w otoczeniu słów ze zbioru PS i jest dodawany do zbioru NS
    \item Następnie dokonywane jest wyszukiwanie w zbiorze danych, takich rekordów, które zawierają słowa ze zbioru NS
    \item Dla każdego znalezionego rekordu w otoczeniu słowa ze zbioru NS wyznaczany jest 1-gram,2-gram oraz 3-gram.
    \item Ze stworzonych n-gramow wybierany jest ten, który występuje najczęściej dla rekordów sarkastycznych w otoczeniu słów ze zbioru NS i jest dodawany do zbioru PS
    \item Powrót do kroku 1, aż do momentu, gdy zbiory PS i NS będą wystarczająco liczne.
\end{enumerate}




Dzięki takiemu podejście możliwe jest zidentyfikowanie najczęściej pojawiających się kombinacji wskazujących na występowanie sarkazmu i w oparciu o nie dokonać predykcji czy dana wypowiedź ma charakter sarkastyczny. Istotną wadą takiego podejścia jest dość wąskie okno w ramach dokonywana jest analiza zdania, a także brak uwzględniania kulturowego znaczenia poszczególnych słów.

Ze względu na obserwowaną lepszą skuteczność klasyfikacji, późniejsze prace skupiały się przede wszystkim na wykorzystaniu sieci neuronowych do zagadnienia rozpoznawania ironii.

Kluczowym elementem w przypadku wykorzystania sieci neuronowych jest wstępne przetworzenie danych. Dlatego prace \cite{Baziotis2018}  \cite{Huang2017}  \cite{Ilic2018} z tej dziedziny skupiały się w pierwszej kolejności na usunięciu lub zastąpieniu specjalnymi oznaczeniami takich informacji jak linki do zewnętrznych portali oraz oznaczenie użytkownika. Ponadto tekst często był normalizowany poprzez wykorzystanie lematyzacji i stemmingu co miało na celu zmniejszenie liczby unikalnych słów. 

Kolejnym istotnym elementem w ramach przetwarzania tekstu, wykorzystującego sieci neuronowe, był sposób konwersji słów do przestrzeni liczbowej. W publikacjach autorzy wykorzystywali różne metody, między innymi powszechnie znany word2vec, czy też zyskujący coraz większa popularność ELMo. Tak przetworzone dane były wprowadzane do sieci zarówno opartych o warstwy konwolucyjne, jak i warstwy typu LSTM. Jakość klasyfikacji dla różnych architekrur opartych o te warstwy była dość podobna z nieznaczną przewagą dla sieci opartych na warstwach LSTM.




\newpage % Rozdziały zaczynamy od nowej strony.
\section{\label{embeddings}Reprezentacja wektorowa słów}

W ramach tego rozdziału zostanie omówione pojęcie reprezentacji wektorowej słowa (z ang. embedding). Ponadto zostaną w nim opisane różne metody uzyskania tych reprezentacji, tak by niosły one w sobie jak najwięcej informacji na temat znaczenia słowa. Rozdział kończy się omówieniem sposobu uwzględnienia informacji na temat cech morfosyntaktycznych słowa w ramach reprezentacji wektorowej, tak by informacja ta mogła być interpretowalna przez sieć neuronową. 

\subsection{Czym jest reprezentacja wektorowa słów}
% zrodło https://towardsdatascience.com/neural-network-embeddings-explained-4d028e6f0526
% zrodlo 2 https://www.tensorflow.org/tutorials/text/word_embeddings
% -----  https://developers.google.com/machine-learning/crash-course/embeddings/video-lecture

% ---- https://en.wikipedia.org/wiki/Word_embedding

% https://pathmind.com/wiki/glossary#cosine

% https://machinelearningmastery.com/what-are-word-embeddings/


Embedding słów jest to zbiorcza nazwa na techniki i narzędzia wykorzystane w ramach przetwarzania języka naturalnego pozwalające na dokonanie mapowania słów, ze zbioru znanych pojęć, na wektor liczb rzeczywistych. Wektor ten, w ramach pracy będzie nazywany reprezentacją wektorową słowa. Z matematycznego punktu widzenia sprowadza się to do transformacji z dyskretnej przestrzeni o wielu wymiarach do ciągłej przestrzeni ze znacznie mniejszą liczbą wymiarów. \cite{Emb_def}
% zrodlo ----> https://en.wikipedia.org/wiki/Word_embedding

%Word embedding is the collective name for a set of language modeling and feature learning techniques in natural language processing (NLP) where words or phrases from the vocabulary are mapped to vectors of real numbers. Conceptually it involves a mathematical embedding from a space with many dimensions per word to a continuous vector space with a much lower dimension.


Takie podejście powoduje, że możliwe jest badanie podobieństwa słów wykorzystując cosinusowe podobieństwo. Jeśli dwa słowa mają podobne znaczenia lub wykorzystywane są w podobnym kontekście  wartość podobieństwa będzie bliższa jednemu. Jeśli słowa rzadko ze sobą występują wartość podobieństwa będzie bliższa zeru.  \cite{Cooos_def}

%todo: można dodać wzór na liczenie...



%zrodlo -> https://www.tensorflow.org/tutorials/text/word_embeddings

%Word embeddings give us a way to use an efficient, dense representation in which similar words have a similar encodin



\subsection{Wykorzystane reprezentacje wektorowe słów}
W ramach pracy są wykorzystywane trzy rodzaje reprezentacji wektorowych słów, każdy z nich różni się długością wektora i sposobem jego uzyskania. Wykorzystane w pracy reprezentacje wektorowe to:
\begin{itemize}
    % \item Word2vec
    \item Glove
    \item FastText
    \item ELMo
\end{itemize}



\subsubsection{Word2vec}
Metoda ta nie została wykorzystana w pracy, jednak jest ona kluczowa do zrzumienia podstaw działania i tworzenia reprezentacji wektorowej słowa dlatego zostanie omówiona w tej sekcji. Występuje w dwóch odmianach:

\begin{itemize}
    \item CBOW
    \item Skip grams
\end{itemize}



\noindent Tworzenie reprezentacji wektorowej słowa w oparciu o model typu CBOW polega na trenowaniu prostej sieci neuronowej. Sieć ta składa się z warstwy wejściowej, jednej warstwy ukrytej (warstwy gęstej) oraz warstwy wyjściowej. Trening modelu polega na podawaniu na wejście słów mieszczących się w ramach pewnego założonego okna, przewidując słowo będące w środku tego okna. Jako okno rozumie się tutaj stałą liczbę słów przed i po aktualnie przewidywanym słowem. Słowa, by mogły być przewidywane przez model, są wcześniej konwertowane do przestrzeni wektorowej poprzez wykorzystanie one-hot encodingu. Proces uczenia polega na systematycznym przesuwaniu okna o jedno słowo i trenowanie wag w warstwie ukrytej. Po zakończonym procesie uczenia należy usunąć ostatnią warstwę wyjściową, a wagi z warstwy ukrytej,wykorzystać do uzyskanie reprezentacji wektorowej słów.\cite{Mikolov2013}

Odmiana ‘skip grams’ jest analogiczna do metody CBOW, tylko zamiast przewidywania jednego słowa w kontekście jego otoczenia, przewidywane jest otoczenie w kontekście słowa.


\subsubsection{Glove}
% https://nlp.stanford.edu/pubs/glove.pdf

Metoda opiera na wykorzystaniu globalnych statystyk współwystępowania słów\cite[]{Pennington2014}, zamiast wykorzystywania tylko informacji o lokalnym kontekście, jak jest to czynione w ramach metody word2vec. Współwystępowanie słów jest zliczane w ramach stałego okna, w ramach które wchodzi skad 10 słów z lewej i z prawej strony.

Trenowanie modelu polega na wyznaczeniu takich reprezentacji wektorowej słów by ich iloczyn skalarny był równy logarytmowi prawdopodobieństwa współwystępowania tych słów w korpusie. Jest to opisane poniższym wzorem:

%todo
% //wstwaić wzór i opisać jego elementy
% \colorbox{yellow}{todo:}\\

$$w_i^T \tilde{w_k}   = log(P_{ik}) = log(X_{ik}) - log(X_i)$$
gdzie: \\
$w_i$ - reprezentacja wektorowa słowa \textbf{i} \\
$w_k$ - reprezentacja wektorowa słowa  \textbf{k} \\
$X_{ik}$ - liczba razy, gdy słowo \textbf{k} występuje w kontekście słowa \textbf{i} \\
$P_{ik} = P(k|i) = X_{ik}/X_i $- prawdopodobieństwo, że słowo \textbf{k} wystąpi w kontekście słowa \textbf{i}\\
$X_{i} = \sum_{k} X_{ik} $ - liczba razy, gdy jakiekolwiek słowo pojawiło się w kontekście słowa  \textbf{i} \\



% \begin{figure}[!h]
%     \label{fig:wzor}
%     \centering \includegraphics[width=0.5\linewidth]{Selection_016.png}
%     \caption{Wzór: //todo: do przeniesiania dolatexa}
% \end{figure}

\subsubsection{FastText}

FastText został stworzony w ramach badań prowadzonych przez firmę Facebook\cite{Bojanowski2016}. Jest on wariacją metody word2vec, która rozważa zdanie na jeszcze niższym poziomie. Metoda ta rozbija słowa na jeszcze mniejsze fragmenty, tzw. n-gramy. Dla przykładu, angielskie słowo ‘where’ (dla okna o wielkości n=3) jest rozbijane na następujące części: <wh, whe, her, ere, re>. Badacze dodali także specjalne symbole ‘<’ oraz ‘>’ pozwalające poprawnie identyfikować przedrostki i przyrostki. Także by zachować informację o pełnej formie słowa, słowo to jest także dodane do zbioru n-gramów.

Tak uzyskana interpretacja słowa jest konwertowana do postaci numerycznego wektora. Polega to na dodawaniu wektorów kolejnych elementów zbioru n-gramów do jednego długiego wektora. Tak uzyskane wektory poszczególnych słów są podawane analogicznie jak w przypadku tradycyjnego word2vec na wejścia płaskiej sieci neuronowej pozwalając finalnie wyznaczyć reprezentacje wektorową słów.

% https://arxiv.org/pdf/1607.04606.pdf
% Enriching Word Vectors with Subword Information

\subsubsection{ELMo}

Ta metoda tworzenia wektorowej reprezentacji słów nie korzysta już z bezpośrednio z podwalin stworzonych przez metodę word2vec. Zamiast tego do wychwytywania kontekstu słowa, niezbędnego do stworzenia poprawnych wektorów, wykorzystuje warstwy LSTM \cite{Gardner2017AllenNLP}. Warstwa ta dzięki swojej budowie pozwala na efektywne zapamiętywanie istotnych informacji, z punktu widzenia zagadnienia klasyfikacji, pojawiających się wcześniej w ramach sekwencji danych wejściowych. Więcej infrmacji na temat sieci LSTM zawartych jest w rozdziale \ref{lstm_subsection}.
%todo
% //wiecej szczeŋółów rozdział o wartwaych  ->odnośnik



\noindent Przetwarzanie zdania wejściowego odbywa się zgodnie z następującymi krokami:

\begin{itemize}
    \item Zdanie jest dzielone na słowa
    \item Każde słowo, przy wykorzystaniu sieci pomocniczej typu CNN(opierającej się o warswy konwolucyjne), ma przypisywany reprezentację wektorową. Jest ona tworzona poprzez agregację wektorowej reprezentacji pojedynczych znaków występujących w ramach słowa. Tak uzyskany wektor jest pierwszym elementem finalnej reprezentacji wektorowej (i jest oznaczony jako $E_1$)
    \item Uzyskany wektor jest podawany na wejście pierwszej warstwy Bi-LSTM. Warstwa ta pozwala na analizę zdania zarówno od początku do końca, jak i od końca do początku.
    \item  Stany ukryte z pierwszej warstwy dla każdego słowa są agregowane w jeden wektor i stanowią drugi element finalnej reprezentacji wektorowej (oznaczony jako $E_2$)
    \item Następnie dane o stanie ukrytym pierwszej warstwy są przekazywane na drugą warstwę Bi-LSTM
    \item Stany ukryte z drugiej warstwy dla każdego słowa są agregowane w jeden wektor i stanowią drugi element finalnej reprezentacji wektorowej (oznaczony jako $E_3$)
    \item W ostaniem kroku następuje połącznie uzyskanych fragmentów finalnej reprezentacji wektorowej, oznaczonych jako $E_1,E_2, E_3$ w jeden wektor o standardowym wymiarze 1024
\end{itemize}



% //todo: dodać rysunek arch -> bo jest duża fajnych 


%todo: opisać czym są postagi
% \subsection{Uwzględnienie POS tagów w ramach embeddingu}
% \subsection{Uwzględnienie, w ramach reprezentacji wektorowej słowa, informacji na temat części mowy, do której słowo należy, wzbogaconej o informację na temat cech morfologicznych słowa. }
\subsection{Uwzględnienie cech morfosyntaktycznych w ramach reprezentacji wektorowej słowa}


% POS tag (Part-of-speach tag) zawiera informacje na temat cech morfosyntatycznych słowa \cite{postags_def}. 

Etykieta morfosyntaktyczna (z ang. Part-of-speach tag) posiada w sobie zbiór inforamcji na temat cech morfosyntaktycznych słowa \cite{postags_def}. Najbardziej podstawową informacją w niej zawartą jest część mowy do której należy słowo, przykładem części mowy możę być między innymi czasownik, rzeczownik oraz przymiotnik. Jednak nie jest to jedyna informacja zawarta w ramach tej etykiety, przechowuje on bowiem także informację o czasie w jakim jest użyty czasownik, w której liczbie jest wykorzystany rzeczownik i wiele innych.

Szczegółowa lista najpopularniejszych klas opisujących cechy morfosyntaktyczne zawarta jest w tabeli \ref{tab:klasy_morf}.


\begin{table}[!h] \centering
    \caption{Tabela przedstawiająca wybrane cechy morfosyntaktyczne przypisane do słowa}
    \label{tab:klasy_morf} 
    \begin{tabular} {| l | l |} \hline
        Oznaczenie klasy & Opis jakie cech morfosyntaktyczne odzwierciedla klasa   \\ \hline\hline
        CC               & spójnik łączny (np. 'and')                              \\ \hline
        CD               & liczba                                                  \\ \hline
        % DT               & określnik                                               \\ \hline
        % EX               & XXXXXX                  \\ \hline
        FW               & słowo obce                                              \\ \hline
        IN               & przyimek,spójnik podrzędny                              \\ \hline
        JJ               & przymiotnik w stopniu równym                            \\ \hline
        JJR              & przymiotnik w stopniu wyższym                           \\ \hline
        JJS              & przymiotnik w stopniu najwyższym                        \\ \hline
        % LS               & XXXXXX                           \\ \hline
        MD               & czasownik modalny                                       \\ \hline
        NN               & rzeczownik w liczbie pojedynczej                        \\ \hline
        NNS              & rzeczownik w liczbie mnogiej                            \\ \hline
        NNP              & rzeczownik w formie nazwy własnej w liczbie pojedynczej \\ \hline
        NNPS             & rzeczownik w formie nazwy własnej w liczbie mnogiej     \\ \hline
        % PDT              & XXXXXX                                                  \\ \hline
        % POS              & XXXXXX                                                  \\ \hline
        PRP              & zaimek osobowy                                          \\ \hline
        PRP              & zaimek dzierżawczy                                      \\ \hline
        RB               & przysłówek w stopniu równym                             \\ \hline
        RBR              & przysłówek w stopniu wyższym                            \\ \hline
        RBS              & przysłówek w stopniu najwyższym                         \\ \hline
        % RP               & XXXXXX                                                  \\ \hline
        % TO               & XXXXXX                                                  \\ \hline
        % UH               & XXXXXX                                                  \\ \hline
        VB               & czasownik w formie podstawowej                          \\ \hline
        VBD              & czasownik w formie przeszłej                            \\ \hline
        VBG              & czasownik w formie imiesłowa czasu teraźniejszego       \\ \hline
        VBN              & czasownik w formie imiesłowa czasu przeszłego           \\ \hline
        VBP              & czasownik w liczbie pojedynczej w formie nie 3-osobowej \\ \hline
        VBZ              & czasownik w liczbie pojedynczej w formie 3-osobowej     \\ \hline
        % WDT              & XXXXXX                                                  \\ \hline
        % WP               & XXXXXX                                                  \\ \hline
        % WP               & XXXXXX                                                  \\ \hline
        % WRB              & XXXXXX                                                  \\ \hline
    \end{tabular}
\end{table}


%todo:
% \colorbox{yellow}{todo:}\\
% //todo: można by dorzucić pełną listę, ale to może być problematyczne

% tu jest lista: https://medium.com/@gianpaul.r/tokenization-and-parts-of-speech-pos-tagging-in-pythons-nltk-library-2d30f70af13b

% https://pythonprogramming.net/natural-language-toolkit-nltk-part-speech-tagging/

Aby umieścić informacje na temat cech morfosyntaktycznych słowa (klasy reprezentującej cechy morfosyntaktyczne slowa) tak by były one możliwe do interpretacji w ramach reprezentacji wektorowej słowa konieczne jest konwertowanie jej do postaci liczbowej. Ze względu na brak dużego podobieństwa między poszczególnymi częściami mowy zdecydowano się na zakodowanie tej informacji w formie one-hot encoding. Tak utworzony wektor był doklejany do istniejącej już reprezentacji wektorowej słowa uzyskanego innymi metodami tworząc wektor dłuższy o 46 elementów.

%todo:
% //todo: dodać rysunek to obrazujący
% \colorbox{yellow}{todo:}\\


\begin{figure}[!h]
    \label{fig:post_tag}
    \centering \includegraphics[width=1\linewidth]{postag_2.png}
    \caption{Przykład reprezentacji wektorowej słowa wraz z zakodowaną informacją na temat cech  morfosyntaktycznych}
\end{figure}
\input{tex/3-dane}
\newpage % Rozdziały zaczynamy od nowej strony.
\section{Preprocessing}

Aby zapewnić jak najlepszą jakość klasyfikacji konieczne jest obrobienie danych do postaci pozbawionej zbędnych szumów. Na potrzeby tej pracy zostały podjęte następujące kroki obróbki danych:

%MOżna napisać czyimi pracami się sugerowałem pisząc preprocessing

\begin{enumerate}
    \item Zastąpienie linków tagiem “<url>”.
    \item Usunięcie znaków końca linii.
    \item Zastąpienie wystąpień nazw użytkowników tagiem  “<username>”.
    \item Zamiana emotikon z formatu “:XX\_YY:” na format “<emote> XX YY” .
    \item Rozbicie hasztagów na mniejsze słowa, oznaczenie go tagiem “<hashtag>”.
    \item Zastąpienie oznaczeń odnoszących się do czasu tagiem “<time>”.
    \item Zastąpienie liczb tagiem “<number>”.
    \item Usunięcie znaków  '' | ''  oraz wielokropków.
    \item Konwersja do małych liter.
    \item Rozwinięcie form skróconych do ich pełnych form.
\end{enumerate}

\begin{table}[!h] \label{tab:tabela2} \centering
    \caption{Tabela przedstawiająca efekt preprocessingu.}
    \begin{tabular} {| c | c | c |} \hline
        Numer operacji preprocessingu & Przed konwersją                                          & Po konwersji                   \\ \hline\hline
        1                             & www.google.pl                                            & <url>                          \\ \hline
        2                             & \textbackslash n \quad \textbackslash r \textbackslash n &                                \\ \hline
        3                             & @mike                                                    & <username>                     \\ \hline
        4                             & :face\_screaming\_in\_fear:                              & <emote> face screaming in fear \\ \hline
        5                             & \#trueFriend                                             & <hashtag> true Friend          \\ \hline
        6                             & 14:58                                                    & <time>                         \\ \hline
        7                             & 1234                                                     & <number>                       \\ \hline
        8                             & | ...                                                    & .                              \\ \hline
        9                             & Kazimierz Wielki                                         & kazimierz wielki               \\ \hline
        10                            & I'll \quad | \quad He's                                  & I will \quad | \quad He has    \\ \hline
    \end{tabular}
\end{table}

%todo:
% być może warto opisać w jaki sposób robię te operacje usuwania ?? -> że regexxy
% być może warto opisać o tych formach skróconych, że to model dodakowy wychwytuje
% można teżnapisać czemu nie wykorzystuję stremingu i lematyzacji
%



\newpage % Rozdziały zaczynamy od nowej strony.
\section{Warstwy}

\subsection{DENSE}
%todo:
// todo: opisać
\subsection{CNN}

% https://towardsdatascience.com/text-classification-rnns-or-cnn-s-98c86a0dd361
% https://towardsdatascience.com/a-comprehensive-guide-to-convolutional-neural-networks-the-eli5-way-3bd2b1164a53

% http://www.wildml.com/2015/11/understanding-convolutional-neural-networks-for-nlp/

Sieci używające CNN z reguły wykorzystywane są w ramach analizy obrazu. Działają one w oparciu o różne filtry przemieszczające się systematycznie po macierzy obrazu. Dany obraz może być jednocześnie analizowany przez wiele filtrów przemieszczających się z różnym skokiem. Pozwala to na naukę rozpoznawania różnych cech obrazu umożliwiających zagadnienia klasyfikacji. Kolejne warstwy CNN można ze sobą łączyć pozwalając na rozpoznawanie coraz bardziej zaawansowanych cech obrazu. 

W przypadku analizy tekstu macierzą wejściową dla warstwy CNN jest macierz o wymiarze AxB, gdzie A to liczba słów w zdaniu, a B to rozmiar embeddingu. Natomiast, ze względu na strukturę danych, sama ekstrakcja cech dokonywana przez filtr może przebiegać tylko w ramach wymiaru A, gdyż dane wzdłuż wymiaru B reprezentują unikalne cechy danego słowa i nie powinny być agregowane. 

%todo: opsiać max pooling ?
//todo: opsiać max pooling ? 

\subsection{RNN}

% http://colah.github.io/posts/2015-08-Understanding-LSTMs/ 

% https://towardsdatascience.com/illustrated-guide-to-recurrent-neural-networks-79e5eb8049c9 


Neuronowa sieć rekurencyjna (RNN) powstała z myślą o przewidywaniu kolejnych elementów sekwencji. Jako wejście przyjmuje kolejne elementy sekwencji, ale w przeciwieństwie do sieci typu feedforward posiada także swój wewnętrzny  stan, w ramach którego zawarta jest informacja o poprzednich wejściach. Sieć tego typu charakteryzuje się jednak bardzo istotną wadą, mianowicie boryka się ona z problem zanikającego gradientu. Jest on spowodowany naturą propagacji wstecznej i wiąże się z coraz mniejszym wpływem początkowych elementów sekwencji na wyjście modelu. Aby temu zaradzić zostały wprowadzone nowe architektury warstw, jedną z nich jest warstwa typu LSTM. 

\subsection{LSTM}

% https://towardsdatascience.com/illustrated-guide-to-lstms-and-gru-s-a-step-by-step-explanation-44e9eb85bf21

Warstwa typu LSTM składa się nuronów zawierających w sobie wiele pośrednich operacji. Poniżej zostaną omówione najważniejsze z nich.

\begin{figure}[!h]
    \label{fig:lstm_diagram}
    \centering \includegraphics[width=1\linewidth]{lstm_diagram.png}
    \caption{Przepływ danych w jedynczym neuronie LSTM}
\end{figure}


\begin{figure}[!h]
    \label{fig:lstm_gates}
    \centering \includegraphics[width=0.5\linewidth]{lstm_gates.png}
    \caption{Przepływ danych w jedynczym neuronie LSTM}
\end{figure}


\subsubsection{Funkcje aktywacji}

Neuron LSTM wykorzystuje dwie funkcje aktywacji: 

\begin{itemize}
    \item Tahn (funkcja aktywacji typu tangens hiperboliczny) - pozwala ona na normalizowanie wektora tak by wartości zawierały się w ramach przedziału [-1,1] 
    \item Sigmoid (funkcja aktywacji sigmoidalna) - przekształca wartości wektora do wartości z przedziału [0,1]. Istotna z punktu widzenia określania, które elementy wektora powinny zostać zapamiętane (wartość bliżej 1), a które zapomniane (wartość bliżej 0) 
\end{itemize}


\subsubsection{Elementy neuronu}

W skład neuronu LSTM wchodzą następujące elementy: 
\begin{itemize}
    \item Forget gate
    \item Input gate 
    \item Cell state 
    \item Output gate
\end{itemize}
 

\paragraph{Forget gate}  \hfill
%  \break

Zadaniem tej bramy jest decydowanie jaka informacja powinna zostać zachowana przez neuron, a jaka usunięta. Informacje z poprzedniego stanu ukrytego ($h_{t-1}$) i informacje z bieżącego wejścia($x_t$) są przekazywane do sigmoidalnej funkcji aktywacji. Powstała macierz jest oznaczona jako $f_t$. Dla każdej pozycji w macierzy zwracane są wartości z zakresu [0,1]. Im bliżej wartości zero, tym wpływ danej pozycji zanika, im bliżej wartości jeden tym wpływ danej pozycji rośnie. 

 
\paragraph{Input gate}  \hfill

Służy do aktualizacji wewnętrznego stanu neuronu (‘cell state’). Informacje z poprzedniego stanu ukrytego ($h_{t-1}$) i informacje z bieżącego wejścia($x_t$) są przekazywane do sigmoidalnej funkcji aktywacji. Tak jak w przypadku 'forget gate’ uzyskiwana jest macierz o wartościach z zakresu [0,1], która tym razem określa jak bardzo informacja, powstała z połącznia $h_{t-1}$ oraz $x_t$, jest ważna z punktu widzenia zadania klasyfikacji. Wyjście z tej operacji oznaczone jest jako $i_t$. 

 W ramach tej bramy dokonuje się normalizacja informacji zawartych w ramach połączenia $h_{t-1}$ oraz $x_t$, poprzez wykorzystanie tangensa hiperbolicznego. Wyjście z tej operacji oznaczone jest jako $c_t$. 

Kolejnym krokiem jest agregacja informacji z $i_t$ oraz $c_t$ poprzez wykonanie operacji mnożenia macierzy. 

\paragraph{Cell state}  \hfill

Ten element neuronu odpowiada za jego stan wewnętrzny i jest przekazywany między kolejnymi komórkami w warstwie. Stan uzyskany z poprzedniej komórki oznaczony jest jako $c_{t-1}$, a z bieżącej jako $c_t$. W celu uzyskania bieżącego stanu komórki wykorzystywane są informacje zgromadzone w ramach poprzednich bram. W pierwszej kolejności dokonywane jest mnożenie macierzowe $c_{t-1}$ oraz $f_t$, które pozwala na usunięcie elementów macierzy nieistotnych z punktu widzenia sieci. Następnie dokonywana jest operacja dodawania uzyskanej macierzy z macierzą uzyskaną w ramach ‘input gate’. Powoduje to dodanie do stanu neuronu nowych informacji istotnych dla sieci. Stan po tej operacji jest określany zmienną $c_t$. 

 
\paragraph{Output gate}  \hfill

Zadaniem tej bramy jest określenie jak powinien wyglądać stan ukryty przekazywany do kolejnego neuronu. Stan ten zawiera informacje na temat poprzednich wejść istotnych z dla zagadnienia klasyfikacji. W ramach tej bramy dokonywane są trzy operacje. Pierwszą z nich jest podanie zagregowanych informacji z poprzedniego stanu ukrytego ($h_{t-1}$) i informacje z bieżącego wejścia($x_t$) na sigmoidalną funkcję aktywacji. W ramach drugiej operacji wewnętrzy stan neuronu (‘cell state’) jest normalizowany przy wykorzystaniu tangesna hiperbolicznego. Trzecią operacją jest mnożenie macierzowe wyników dwóch poprzednich operacji, w efekcie uzyskiwany jest stan ukryty neuronu, oznaczony jako $h_t$, przekazywany do następnego neuronu.

\subsection{Bi-LSTM}

%https://medium.com/@raghavaggarwal0089/bi-lstm-bc3d68da8bd0 

Warstwa składa się z dwóch tak samo zbudowanych warstw LSTM. Jedna z warstw przetwarza ciąg danych wejściowych od początku do końca, a druga od końca do początku. Pozwala to na naukę zależności między elementami w dwóch kierunkach, co przekłada się na większą liczbę informacji na temat sekwencji, co pozwala na szybsze i dokładniejsze trenowanie modelu.  

 



 
\newpage % Rozdziały zaczynamy od nowej strony.
\section{Metryki}

Jakość klasyfikacji można monitorować przy wykorzystaniu różnych metryk. Najbardziej popularną metryką jest dokładność (ang. accuraccy), czyli stosunek poprawnie zakwalifikowanych przez model obserwacji, do całkowitej liczby obserwacji w zbiorze. Nie jest to jednak miara idealna, nie bierze ona pod uwagę między innymi tego czy badany zbiór jest zbilansowany. W przypadku zbioru niezbilansowanego możliwe jest uzyskanie bardzo dobrej dokładności nawet przy zakwalifikowaniu wszystkich obserwacji do tylko jednej klasy. W celu wyeliminowania tego problemu wykorzystuje się inne metryki, takie jak precyzja i czułość.



Powyższe metryki wykorzystują następujące pojęcia pomocnicze:
\begin{itemize}
    \item true positive (TP) - liczba pozytywnych obserwacji zaklasyfikowanych jako pozytywne
    \item true negative (TN) - liczba negatywnych obserwacji zaklasyfikowanych jako negatywne
    \item false positive (FP) - liczba negatywnych obserwacji zaklasyfikowanych jako pozytywne
    \item false negative (FN) - liczba pozytywnych obserwacji zaklasyfikowanych jako negatywne
\end{itemize}


W oparciu o te pojęcia pomocnicze można sformułować następujące zależności:

% $$
%     dokładność = \frac{\ liczba\ poprawnie\ zakwalifikowanych\ obserwacji\ }{\ całkowita\ liczba\ obserwacji\ }
% $$

$$
    precyzja = \frac{TP}{TP + FP}
$$

$$
    czułość = \frac{TP}{TP + FN}
$$


%https://developers.google.com/machine-learning/crash-course/classification/precision-and-recall 

%https://blog.exsilio.com/all/accuracy-precision-recall-f1-score-interpretation-of-performance-measures/ 

%https://towardsdatascience.com/accuracy-precision-recall-or-f1-331fb37c5cb9 

W oparciu o wzory można zauważyć, że precyzja zawiera w sobie informację jaka część obserwacji zakwalifikowanych jako pozytywne była zakwalifikowana poprawnie. Natomiast czułość zawiera informację na temat tego jaka część pozytywnych obserwacji została zakwalifikowana poprawnie.

Precyzja i czułość są ze sobą współzależne, w przypadku, gdy wskazania jednej z metryk się poprawiają, to wskazania drugiej metryki spadają. Aby monitować optymalne wskazania obu metryk powstała metryka F1, jest ona opisana poniższym wzorem:

$$
    F1 = 2 \cdot \frac{precyzja \cdot czułość}{precyzja + czułość}
$$




% $\lim_{n \to \infty}
%     \sum_{k=1}^n \frac{1}{k^2}
%     = \frac{\pi^2}{6}$
% \\ \\
% $$
%     \lim_{n \to \infty}
%     \sum_{k=1}^n \frac{1}{k^2}
%     = \frac{\pi^2}{6}
% $$





\newpage % Rozdziały zaczynamy od nowej strony.
\section{Eksperymenty}

W ramach pracy została dokonana analiza wpływu oznaczeń morfosyntaktycznych na jakość klasyfikacji tekstów ironicznych. Na potrzeby języka angielskiego taka analiza sprowadza się głownie do analizy wpływu typu części mowy do jakiego należą poszczególne słowa i taka analiza zostanie podjęta w kolejnych krokach.

% \colorbox{yellow}{highlight}\\

Na potrzeby analizy wykorzystano wiele różnych modeli sieci neuronowych, bazujących głownie na trzech typach warstw:

\begin{itemize}
    \item W pełni połączonych;
    \item CNN;
    \item LSTM;
\end{itemize}

W ramach różnych architektur testowany był wpływ kilku hiperparametrów na jakość klasyfikacji. Oto lista niektórych z nich:
\begin{itemize}
    \item W pełni połączonych:
          \begin{itemize}
              \item  1/2/3 warstwy
              \item Z dropoutem / bez dropoutu
              \item Wielkość wartwy ukrytej 30-200
          \end{itemize}

    \item CNN;
          \begin{itemize}
              \item  1/2/3 warstwy
              \item Wielkość okna 3-15
              \item Z dropoutem / bez dropoutu
              \item Wielkość wartwy ukrytej 30-200
          \end{itemize}
    \item LSTM;
          \begin{itemize}
              \item Wykorzystanie warsty pojedynczej (LSTM) oraz podwójnej (Bi-LSTM)
              \item  1/2/3 warstwy
              \item Z dropoutem / bez dropoutu
              \item Wielkość wartwy ukrytej 30-200
          \end{itemize}
\end{itemize}

\subsection{POSTAGS}
%todo:
//todo: zmergować z dugim rozdiałem o postagach

Aby możliwe było wykorzystanie przez sieć neuronową informacji na temat typu części mowy danego słowa konieczne są następujące kroki:

\begin{enumerate}
    \item Rozpoznanie części mowy w kontekście zdania i przypisanie do niego odpowiedniego tagu,
    \item Konwersja wykrytego tagu do liczbowej reprezentacji wektorowej,
    \item Dołączenie utworzonego wektora do przetrenowanego wektora embedingów.
\end{enumerate}

\hfill \break % to dodaje pustą linie po czymść dziwnym -lista, tabela
Do rozpoznania części mowy zostało wykorzystane narzędzie z pakietu NLTK, które pozwala na oznaczenie każdego słowa w zdaniu jednym z 46 tagów, każdy odpowiadający innej części mowy. Następnie wykryty tag został przetworzony do postaci “one-hot encoding” i dodany do istniejącego embedingu słowa, zwiększając tym samy jego rozmiar o długość 46.


\subsection{Wyniki}

Aby w pełni zweryfikować wpływ części mowy (POS tagów) na jakość klasyfikacji zostały wykorzystane trzy różne embeddingi oraz dwa różne zbiory danych wejściowych. Embeddingi różnią się one między innymi długością oraz metodą ich uzyskiwania. A zbiory danych różnią się licznością i stopniem zaszumienia. Każdy ze zbiorów danych został podzielony na trzy zbiory treningowy, walidacyjny oraz testowy w proporcjach 60/20/20 procent.

Poniżej przedstawiono tabele z wynikami obliczeń.

%todo:
\colorbox{yellow}{todo:}\\

% \begin{table}[!h] \label{tab:wyniki_XXX} \centering
%     \begin{tabular} {| c | c | r |} \hline
%         Sieć  & Accuracy & Liczba \\ \hline\hline
%               & Accuracy & Liczba \\ \hline\hline
%         cell1 & cell2    & 60     \\ \hline
%         cell4 & cell5    & 43     \\ \hline
%         cell7 & cell8    & 20,45  \\ \hline
%     \end{tabular}
% \end{table}

%todo:
\colorbox{yellow}{todo: poprawić nawy i dodać poróœnanie wartośći w tabelkach}\\

\begin{table}[!h] \label{tab:wyniki_glove_A} \centering
    \caption{Embedding Glove o długości 25, zbiór A}

    \begin{tabular} {|c|c|c|c|c|c|c|c|c| }    \hline
                                          & \multicolumn{8}{c|}{metryki}                                                                                                                               \\ \hline
        X                                 & \multicolumn{2}{c|}{accuracy} & \multicolumn{2}{c|}{precision} & \multicolumn{2}{c|}{recall} & \multicolumn{2}{c|}{f1}                                     \\ \hline
        sieć                              & N                             & P                              & N                           & P                       & N      & P      & N      & P      \\ \hline
        give\_model\_00                   & 0.6327                        & 0.6372                         & 0.6074                      & 0.6210                  & 0.6853 & 0.6456 & 0.6440 & 0.6331 \\ \hline
        give\_model\_00\_1                & 0.6022                        & 0.5943                         & 0.5790                      & 0.5732                  & 0.6573 & 0.6386 & 0.6157 & 0.6041 \\ \hline
        give\_model\_00\_2                & 0.6192                        & 0.6079                         & 0.5946                      & 0.5748                  & 0.6736 & 0.7342 & 0.6316 & 0.6448 \\ \hline
        give\_model\_00\_3                & 0.6395                        & 0.6169                         & 0.6095                      & 0.5978                  & 0.7132 & 0.6410 & 0.6573 & 0.6186 \\ \hline
        give\_model\_10                   & 0.6440                        & 0.6677                         & 0.6446                      & 0.6261                  & 0.5920 & 0.7808 & 0.6172 & 0.6950 \\ \hline
        give\_model\_10\_1                & 0.6655                        & 0.6271                         & 0.6394                      & 0.5932                  & 0.7109 & 0.7342 & 0.6732 & 0.6562 \\ \hline
        give\_model\_10\_2                & 0.6632                        & 0.6598                         & 0.6276                      & 0.6624                  & 0.7505 & 0.6083 & 0.6836 & 0.6342 \\ \hline
        give\_model\_10\_3                & 0.6598                        & 0.6644                         & 0.6151                      & 0.6325                  & 0.7972 & 0.7342 & 0.6944 & 0.6796 \\ \hline
        give\_model\_10\_4                & 0.6711                        & 0.6587                         & 0.6347                      & 0.6191                  & 0.7575 & 0.7692 & 0.6907 & 0.6860 \\ \hline
        give\_model\_10\_5                & 0.6677                        & 0.6632                         & 0.6470                      & 0.6541                  & 0.6923 & 0.6480 & 0.6689 & 0.6510 \\ \hline
        give\_model\_10\_6                & 0.6700                        & 0.6531                         & 0.6395                      & 0.6150                  & 0.7319 & 0.7599 & 0.6826 & 0.6798 \\ \hline
        give\_model\_20                   & 0.6429                        & 0.6474                         & 0.6506                      & 0.6291                  & 0.5687 & 0.6643 & 0.6069 & 0.6462 \\ \hline
        give\_model\_30                   & 0.6485                        & 0.6644                         & 0.6006                      & 0.6823                  & 0.8205 & 0.5757 & 0.6935 & 0.6245 \\ \hline
        give\_model\_40                   & 0.6644                        & 0.6531                         & 0.6460                      & 0.6150                  & 0.6806 & 0.7599 & 0.6628 & 0.6798 \\ \hline
        give\_model\_41                   & 0.6361                        & 0.6610                         & 0.6389                      & 0.6267                  & 0.5734 & 0.7435 & 0.6044 & 0.6801 \\ \hline
        give\_model\_50                   & 0.6689                        & 0.6689                         & 0.6497                      & 0.6387                  & 0.6876 & 0.7296 & 0.6681 & 0.6811 \\ \hline
        give\_model\_50000                & 0.6519                        & 0.6497                         & 0.6170                      & 0.6178                  & 0.7435 & 0.7272 & 0.6744 & 0.6680 \\ \hline
        give\_model\_50001                & 0.6610                        & 0.6632                         & 0.6532                      & 0.6367                  & 0.6410 & 0.7109 & 0.6470 & 0.6718 \\ \hline
        give\_model\_60                   & 0.6576                        & 0.6598                         & 0.6211                      & 0.6385                  & 0.7529 & 0.6876 & 0.6807 & 0.6621 \\ \hline
        give\_model\_61                   & 0.6677                        & 0.6677                         & 0.6464                      & 0.6496                  & 0.6946 & 0.6829 & 0.6696 & 0.6659 \\ \hline
        give\_model\_8000\_cnn            & 0.6519                        & 0.6485                         & 0.6416                      & 0.6189                  & 0.6386 & 0.7156 & 0.6401 & 0.6637 \\ \hline
        give\_model\_8001\_cnn            & 0.6316                        & 0.6485                         & 0.6107                      & 0.6365                  & 0.6620 & 0.6410 & 0.6353 & 0.6387 \\ \hline
        give\_model\_8001\_cnn\_15        & 0.6203                        & 0.6384                         & 0.5954                      & 0.6326                  & 0.6759 & 0.6060 & 0.6331 & 0.6190 \\ \hline
        give\_model\_8001\_cnn\_15\_pool  & 0.6440                        & 0.6508                         & 0.6515                      & 0.6345                  & 0.5710 & 0.6596 & 0.6086 & 0.6468 \\ \hline
        give\_model\_8001\_cnn\_15\_pool2 & 0.6327                        & 0.6395                         & 0.6125                      & 0.6200                  & 0.6596 & 0.6620 & 0.6352 & 0.6403 \\ \hline
        give\_model\_8001\_cnn\_2\_layers & 0.6395                        & 0.6598                         & 0.6238                      & 0.6422                  & 0.6456 & 0.6736 & 0.6345 & 0.6575 \\ \hline
        give\_model\_8001\_cnn\_3\_layers & 0.6225                        & 0.6237                         & 0.5975                      & 0.5810                  & 0.6783 & 0.8018 & 0.6353 & 0.6738 \\ \hline
        give\_model\_8001\_cnn\_5         & 0.6305                        & 0.6485                         & 0.6153                      & 0.6347                  & 0.6340 & 0.6480 & 0.6245 & 0.6412 \\ \hline
        give\_model\_8001\_cnn\_7         & 0.6406                        & 0.6338                         & 0.6241                      & 0.6143                  & 0.6503 & 0.6573 & 0.6369 & 0.6351 \\ \hline
        give\_model\_8001\_cnn\_9         & 0.6293                        & 0.6440                         & 0.6150                      & 0.6350                  & 0.6293 & 0.6247 & 0.6221 & 0.6298 \\ \hline
        give\_model\_8002\_cnn            & 0.6395                        & 0.6327                         & 0.6227                      & 0.6111                  & 0.6503 & 0.6666 & 0.6362 & 0.6376 \\ \hline
    \end{tabular}
\end{table}


\begin{table}[!h] \label{tab:wyniki_fastext_A} \centering
    \caption{Embedding FastText o długości 300, zbiór A}
    \begin{tabular}{|c|c|c|c|c|c|c|c|c|}    \hline

                                          & \multicolumn{8}{c|}{metryki}                                                                                                                               \\ \hline
        X                                 & \multicolumn{2}{c|}{accuracy} & \multicolumn{2}{c|}{precision} & \multicolumn{2}{c|}{recall} & \multicolumn{2}{c|}{f1}                                     \\ \hline
        sieć                              & N                             & P                              & N                           & P                       & N      & P      & N      & P      \\ \hline
        give\_model\_00                   & 0.6598                        & 0.6497                         & 0.6355                      & 0.6307                  & 0.6993 & 0.6689 & 0.6659 & 0.6493 \\ \hline
        give\_model\_00\_1                & 0.6677                        & 0.6576                         & 0.6490                      & 0.6334                  & 0.6853 & 0.6969 & 0.6666 & 0.6637 \\ \hline
        give\_model\_00\_2                & 0.6440                        & 0.6327                         & 0.6126                      & 0.6                     & 0.7226 & 0.7272 & 0.6631 & 0.6575 \\ \hline
        give\_model\_00\_3                & 0.6655                        & 0.6418                         & 0.6501                      & 0.6129                  & 0.6713 & 0.7086 & 0.6605 & 0.6572 \\ \hline
        give\_model\_10                   & 0.6813                        & 0.6824                         & 0.6909                      & 0.6439                  & 0.6200 & 0.7715 & 0.6535 & 0.7020 \\ \hline
        give\_model\_10\_1                & 0.6485                        & 0.6655                         & 0.6322                      & 0.6405                  & 0.6573 & 0.7062 & 0.6445 & 0.6718 \\ \hline
        give\_model\_10\_2                & 0.6632                        & 0.6813                         & 0.6219                      & 0.6622                  & 0.7785 & 0.6993 & 0.6915 & 0.6802 \\ \hline
        give\_model\_10\_3                & 0.6779                        & 0.6576                         & 0.6463                      & 0.6184                  & 0.7412 & 0.7668 & 0.6905 & 0.6847 \\ \hline
        give\_model\_10\_4                & 0.6870                        & 0.6677                         & 0.6681                      & 0.6537                  & 0.7039 & 0.6689 & 0.6855 & 0.6612 \\ \hline
        give\_model\_10\_5                & 0.6926                        & 0.6813                         & 0.6780                      & 0.6496                  & 0.6969 & 0.7435 & 0.6873 & 0.6934 \\ \hline
        give\_model\_10\_6                & 0.6813                        & 0.6553                         & 0.6689                      & 0.6152                  & 0.6783 & 0.7715 & 0.6736 & 0.6845 \\ \hline
        give\_model\_20                   & 0.6711                        & 0.6700                         & 0.6431                      & 0.6751                  & 0.7226 & 0.6153 & 0.6805 & 0.6439 \\ \hline
        give\_model\_30                   & 0.6779                        & 0.6497                         & 0.6698                      & 0.6064                  & 0.6620 & 0.7902 & 0.6658 & 0.6862 \\ \hline
        give\_model\_40                   & 0.6836                        & 0.6474                         & 0.6769                      & 0.6338                  & 0.6643 & 0.6456 & 0.6705 & 0.6397 \\ \hline
        give\_model\_41                   & 0.6542                        & 0.6508                         & 0.6242                      & 0.6162                  & 0.7202 & 0.7412 & 0.6688 & 0.6730 \\ \hline
        give\_model\_50                   & 0.6836                        & 0.6768                         & 0.6712                      & 0.6682                  & 0.6806 & 0.6620 & 0.6759 & 0.6651 \\ \hline
        give\_model\_50000                & 0.6779                        & 0.6757                         & 0.6538                      & 0.6748                  & 0.7132 & 0.6386 & 0.6822 & 0.6562 \\ \hline
        give\_model\_50001                & 0.6711                        & 0.6937                         & 0.6526                      & 0.7219                  & 0.6876 & 0.5990 & 0.6696 & 0.6547 \\ \hline
        give\_model\_60                   & 0.6790                        & 0.6508                         & 0.6593                      & 0.6102                  & 0.6993 & 0.7738 & 0.6787 & 0.6824 \\ \hline
        give\_model\_61                   & 0.6824                        & 0.6757                         & 0.6674                      & 0.6961                  & 0.6876 & 0.5874 & 0.6773 & 0.6371 \\ \hline
        give\_model\_8000\_cnn            & 0.6564                        & 0.6384                         & 0.6310                      & 0.6142                  & 0.7016 & 0.6829 & 0.6644 & 0.6467 \\ \hline
        give\_model\_8001\_cnn            & 0.6542                        & 0.6429                         & 0.6413                      & 0.6263                  & 0.6503 & 0.6526 & 0.6458 & 0.6392 \\ \hline
        give\_model\_8001\_cnn\_15        & 0.6271                        & 0.6429                         & 0.6092                      & 0.6287                  & 0.6433 & 0.6433 & 0.6258 & 0.6359 \\ \hline
        give\_model\_8001\_cnn\_15\_pool  & 0.6372                        & 0.6474                         & 0.6310                      & 0.6258                  & 0.6060 & 0.6783 & 0.6183 & 0.6510 \\ \hline
        give\_model\_8001\_cnn\_15\_pool2 & 0.6598                        & 0.6542                         & 0.6428                      & 0.6382                  & 0.6713 & 0.6620 & 0.6567 & 0.6498 \\ \hline
        give\_model\_8001\_cnn\_2\_layers & 0.6485                        & 0.6519                         & 0.6213                      & 0.6257                  & 0.7039 & 0.7016 & 0.6601 & 0.6615 \\ \hline
        give\_model\_8001\_cnn\_3\_layers & 0.6644                        & 0.6644                         & 0.6392                      & 0.6434                  & 0.7062 & 0.6899 & 0.6710 & 0.6659 \\ \hline
        give\_model\_8001\_cnn\_5         & 0.6451                        & 0.6395                         & 0.6220                      & 0.6211                  & 0.6829 & 0.6573 & 0.6511 & 0.6387 \\ \hline
        give\_model\_8001\_cnn\_7         & 0.6372                        & 0.6485                         & 0.6168                      & 0.6288                  & 0.6643 & 0.6713 & 0.6397 & 0.6493 \\ \hline
        give\_model\_8001\_cnn\_9         & 0.6485                        & 0.6485                         & 0.6277                      & 0.6299                  & 0.6759 & 0.6666 & 0.6509 & 0.6477 \\ \hline
        give\_model\_8002\_cnn            & 0.6451                        & 0.6542                         & 0.6321                      & 0.6363                  & 0.6410 & 0.6689 & 0.6365 & 0.6522 \\ \hline
    \end{tabular}
\end{table}


\begin{table}[!h] \label{tab:wyniki_fastext_A} \centering
    \caption{Embedding ELMo o długości 1024, zbiór A}
    \begin{tabular}{|c|c|c|c|c|c|c|c|c|}    \hline
                                          & \multicolumn{8}{c|}{metryki}                                                                                                                               \\ \hline
        X                                 & \multicolumn{2}{c|}{accuracy} & \multicolumn{2}{c|}{precision} & \multicolumn{2}{c|}{recall} & \multicolumn{2}{c|}{f1}                                     \\ \hline
        sieć                              & N                             & P                              & N                           & P                       & N      & P      & N      & P      \\ \hline
        give\_model\_00                   & 0.6531                        & 0.6779                         & 0.6392                      & 0.6487                  & 0.6526 & 0.7319 & 0.6459 & 0.6878 \\ \hline
        give\_model\_00\_1                & 0.6519                        & 0.6429                         & 0.6273                      & 0.6184                  & 0.6946 & 0.6876 & 0.6592 & 0.6512 \\ \hline
        give\_model\_00\_2                & 0.6203                        & 0.6418                         & 0.5882                      & 0.6261                  & 0.7226 & 0.6480 & 0.6485 & 0.6368 \\ \hline
        give\_model\_00\_3                & 0.6406                        & 0.6485                         & 0.6065                      & 0.6520                  & 0.7365 & 0.5897 & 0.6652 & 0.6193 \\ \hline
        give\_model\_10                   & 0.6836                        & 0.6655                         & 0.6795                      & 0.6164                  & 0.6573 & 0.8205 & 0.6682 & 0.704  \\ \hline
        give\_model\_10\_1                & 0.7152                        & 0.6903                         & 0.7015                      & 0.6611                  & 0.7179 & 0.7412 & 0.7096 & 0.6989 \\ \hline
        give\_model\_10\_2                & 0.6926                        & 0.6711                         & 0.6891                      & 0.6443                  & 0.6666 & 0.7179 & 0.6777 & 0.6791 \\ \hline
        give\_model\_10\_3                & 0.6802                        & 0.6892                         & 0.6666                      & 0.6492                  & 0.6806 & 0.7808 & 0.6735 & 0.7089 \\ \hline
        give\_model\_10\_4                & 0.6971                        & 0.6768                         & 0.6687                      & 0.6382                  & 0.7435 & 0.7692 & 0.7041 & 0.6976 \\ \hline
        give\_model\_10\_5                & 0.6926                        & 0.7129                         & 0.6772                      & 0.6826                  & 0.6993 & 0.7622 & 0.6880 & 0.7202 \\ \hline
        give\_model\_10\_6                & 0.7186                        & 0.7118                         & 0.7                         & 0.6986                  & 0.7342 & 0.7132 & 0.7167 & 0.7058 \\ \hline
        give\_model\_20                   & 0.6734                        & 0.6926                         & 0.6933                      & 0.7524                  & 0.5850 & 0.5454 & 0.6346 & 0.6324 \\ \hline
        give\_model\_30                   & 0.6903                        & 0.7050                         & 0.6962                      & 0.7038                  & 0.6410 & 0.6759 & 0.6674 & 0.6896 \\ \hline
        give\_model\_40                   & 0.6892                        & 0.6847                         & 0.6824                      & 0.6820                  & 0.6713 & 0.6550 & 0.6768 & 0.6682 \\ \hline
        give\_model\_41                   & 0.6757                        & 0.6802                         & 0.6658                      & 0.6442                  & 0.6643 & 0.7599 & 0.6651 & 0.6973 \\ \hline
        give\_model\_50                   & 0.6971                        & 0.6813                         & 0.6716                      & 0.6981                  & 0.7342 & 0.6037 & 0.7015 & 0.6475 \\ \hline
        give\_model\_50000                & 0.7039                        & 0.6711                         & 0.6937                      & 0.6699                  & 0.6969 & 0.6340 & 0.6953 & 0.6514 \\ \hline
        give\_model\_50001                & 0.6960                        & 0.6903                         & 0.66                        & 0.7066                  & 0.7692 & 0.6177 & 0.7104 & 0.6592 \\ \hline
        give\_model\_60                   & 0.6949                        & 0.6892                         & 0.6593                      & 0.6816                  & 0.7668 & 0.6736 & 0.7090 & 0.6776 \\ \hline
        give\_model\_61                   & 0.6937                        & 0.7016                         & 0.7078                      & 0.6813                  & 0.6270 & 0.7226 & 0.6650 & 0.7013 \\ \hline
        give\_model\_8000\_cnn            & 0.6282                        & 0.6418                         & 0.6086                      & 0.6296                  & 0.6526 & 0.6340 & 0.6299 & 0.6318 \\ \hline
        give\_model\_8001\_cnn            & 0.6508                        & 0.6531                         & 0.6333                      & 0.6326                  & 0.6643 & 0.6783 & 0.6484 & 0.6546 \\ \hline
        give\_model\_8001\_cnn\_15        & 0.6497                        & 0.6485                         & 0.64                        & 0.6340                  & 0.6340 & 0.6503 & 0.6370 & 0.6421 \\ \hline
        give\_model\_8001\_cnn\_15\_pool  & 0.6564                        & 0.6542                         & 0.6443                      & 0.6394                  & 0.6503 & 0.6573 & 0.6473 & 0.6482 \\ \hline
        give\_model\_8001\_cnn\_15\_pool2 & 0.6598                        & 0.6587                         & 0.6454                      & 0.6420                  & 0.6620 & 0.6689 & 0.6536 & 0.6552 \\ \hline
        give\_model\_8001\_cnn\_2\_layers & 0.6519                        & 0.6203                         & 0.6437                      & 0.6120                  & 0.6317 & 0.5920 & 0.6376 & 0.6018 \\ \hline
        give\_model\_8001\_cnn\_3\_layers & 0.6610                        & 0.6632                         & 0.6592                      & 0.6356                  & 0.6223 & 0.7156 & 0.6402 & 0.6732 \\ \hline
        give\_model\_8001\_cnn\_5         & 0.6451                        & 0.6406                         & 0.6258                      & 0.6158                  & 0.6666 & 0.6876 & 0.6455 & 0.6497 \\ \hline
        give\_model\_8001\_cnn\_7         & 0.6519                        & 0.6384                         & 0.6318                      & 0.6339                  & 0.6759 & 0.6013 & 0.6531 & 0.6172 \\ \hline
        give\_model\_8001\_cnn\_9         & 0.6632                        & 0.6497                         & 0.6427                      & 0.6461                  & 0.6876 & 0.6130 & 0.6644 & 0.6291 \\ \hline
        give\_model\_8002\_cnn            & 0.6293                        & 0.6451                         & 0.5980                      & 0.6125                  & 0.7179 & 0.7296 & 0.6525 & 0.6659 \\ \hline
    \end{tabular}
\end{table}

//tabelka 25 -> zbior2 000/ zbior20 000

//tabelka 300-> zbior2 000/ zbior20 000

//tabelka 1024-> zbior2 000/ zbior20 000

W oparciu o wyniki zamieszczone w powyższych tabelkach można stwierdzić, że wraz z dołączeniem informacji o częściach mowy nie obserwuje się widocznej poprawy jakości klasyfikacji, czasem występuje nawet jej pogorszenie. Jednym z powodów takiego zachowania może być fakt, że rodzaj części mowy nie jest kluczowy do rozpoznania ironicznego charakteru tekstu. Możne to być też spowodowane tym, że sieci wychwytują niezbędne dla nich informacje na temat roli konkretnego słowa w zdaniu w oparciu o uzyskiwany kontekst i dodatkowa informacja w ramach embeddingu jest już niepotrzebna.

Ponadto można zaważyć, że wraz ze wzrostem rozmiaru embedingów poprawia się jakość klasyfikacji, co jest zgodne z oczekiwaniami, jako że dłuższy embedding pozwala na zachowanie większej liczby informacji na temat słowa.


Analizując uśrednione pomiary jakości klasyfikacji dla sieci opartych o różne typy warstw, można zauważyć, że najgorzej sprawują się sieci oparte o warstwy w pełni połączone. Natomiast sieci wykorzystujące CNN oraz LSTM posiadają podobną skuteczność klasyfikacji. Aby wykluczyć wpływ zaszumienia danych oraz przeuczenia sieci wykonano jeszcze dwa eksperymenty mające porównać skuteczność klasyfikacji sieci opartych na warstwach typu LSTM oraz CNN. Eksperymenty te są przeprowadzone na danych wejściowych pozbawionych informacji o częściach mowy, gdyż z dotychczasowych pomiarów wynika, iż mają one pomijalny wpływ na jakość klasyfikacji.

\colorbox{yellow}{todo: todo opisanie, że zbior duży to zbiór B, a mały zbiór to zbiór A}\\
%todo opisanie, że zbior duży to zbiór B, a mały zbiór to zbiór A
\subsection{Uczenie modelu na zbiorze B i testowanie jakoś klasyfikacji na zbiorze A }

W ramach tego eksperymentu zbiór danych oznaczonych jako \textbf{B} został podzielony na dwa zbiory: treningowy oraz w walidacyjny w proporcjach 80/20 procent. Natomiast zbiór oznaczony jako A został w całości wykorzystany jako zbiór testowy. Wyniki zawierają tabele poniżej.

%todo:
\colorbox{yellow}{todo:}\\
//tabela dla różnych embedingów pokazująca, że CNN mają najlepszą zdolnośc do generalizacji

%todo:
\colorbox{yellow}{todo:}\\
//Wymyślić wnioski dla teakiej sytuacji


\subsection{Uczenie i testowanie modelu na fragmencie zbioru dużego B tak by był równoliczny ze zbiorem małym A}

W ramach tego eksperymentu ze zbioru danych oznaczonych jako \textbf{B} zostało wybranych w losowy sposób 2000 rekordów ironicznych i tyle samo nieironicznych, tak by licznością były porównywalne ze zbiorem oznaczonym jako \textbf{A}. Tak uzyskany zbiór został podzielony na trzy podzbiory: treningowy, walidacyjny oraz testowy w proporcjach 60/20/20 procent. Wyniki zawierają tabele poniżej.

%todo:
\colorbox{yellow}{todo:}\\
//tabela dla roznych embedingow pokazujaca, że CNN jest o 5 procent gorsze od LSTM

W oparciu o uzyskane wyniki można zauważyć:

\begin{itemize}
    \item Wyższą średnią jakoś klasyfikacji w porównaniu do zbioru oznaczonego jako \textbf{A}
    \item Lepszą jakość klasyfikacji dla sieci opartych o warstwy typu LSTM
\end{itemize}


Wyższa jakoś klasyfikacji może wskazywać na to, że zbiór oznaczony jako \textbf{A} posiada dużo bardziej zróżnicowane dane, przy czym zbiór posiada zbyt małą liczbą próbek by sieć, w procesie nauki, odpowiednio je wychwyciła. Problem ten może być najprawdopodobniej wyeliminowany poprzez dostarczenie większej liczby próbek danych o zróżnicowanym charakterze.

Lepsza jakość klasyfikacji przez sieci oparte o warstwy LSTM jest zgodny z wnioskami płynącymi z innych publikacji badających podobne zagadnienia. Trudności z obserwacji tej zależności w poprzednich eksperymentach mogła wynikać między innymi z:


\begin{itemize}

    \item Dla zbioru A:
          \begin{itemize}
              \item Małej liczby próbek
              \item Dużej różnorodności prezentowanej ironii w próbkach
          \end{itemize}

    \item Dla zbioru B:
          \begin{itemize}
              \item Małej różnorodności prezentowanej ironii w próbkach
              \item Na tyle duża liczba próbek, by sieci o teoretycznie gorszych zdolnościach do rozwiązywania danego zadania klasyfikacji mogły wychwycić cechy tekstu pozwalające, przy odpowiednio długim procesie uczenia, na poprawną klasyfikację.
          \end{itemize}

\end{itemize}







\newpage % Rozdziały zaczynamy od nowej strony.
\subsection{Wyniki eksperymentów} \label{wyniki_eksperymentow}

W ramach tego rozdziału umieszczone są dane dotyczące jakości klasyfikacji poszczególnych modeli w ramach wykonanych eksperymentów.

\subsubsection{Wyniki dla eksperymentu 1 - zbiór A}

%todo:
\colorbox{yellow}{todo: dodać poróœnanie wartośći w tabelkach}

\begin{table}[h] \centering
    \caption{Embedding Glove o długości 25, zbiór A}
    \label{tab:wyniki_glove_A}
    \begin{tabular} {|c|c|c|c|c|c|c|c|c| }    \hline
                 & \multicolumn{8}{c|}{metryki}                                                                                                                                                                                                                                                                                  \\ \hline
        X        & \multicolumn{2}{c|}{accuracy}       & \multicolumn{2}{c|}{precision}      & \multicolumn{2}{c|}{recall}         & \multicolumn{2}{c|}{f1}                                                                                                                                                                     \\ \hline
        sieć     & N                                   & P                                   & N                                   & P                                   & N                                   & P                                   & N                                   & P                                   \\ \hline
        dense\_1 & 0.6327                              & \textbf{0.6372}                     & 0.6074                              & \textbf{0.6210}                     & \textbf{0.6853}                     & 0.6456                              & \textbf{0.6440}                     & 0.6331                              \\ \hline
        dense\_2 & \textbf{0.6022}                     & 0.5943                              & \textbf{0.5790}                     & 0.5732                              & \textbf{0.6573}                     & 0.6386                              & \textbf{0.6157}                     & 0.6041                              \\ \hline
        dense\_3 & \textbf{0.6192}                     & 0.6079                              & \textbf{0.5946}                     & 0.5748                              & 0.6736                              & \textbf{0.7342}                     & 0.6316                              & \textbf{0.6448}                     \\ \hline
        dense\_4 & \textbf{0.6395}                     & 0.6169                              & \textbf{0.6095}                     & 0.5978                              & \textbf{0.7132}                     & 0.6410                              & \textbf{0.6573}                     & 0.6186                              \\ \hline
        lstm\_1  & 0.6440                              & \textbf{0.6677}                     & \textbf{0.6446}                     & 0.6261                              & 0.5920                              & \textbf{0.7808}                     & 0.6172                              & \textbf{\cellcolor{green!50}0.6950} \\ \hline
        lstm\_2  & \textbf{0.6655}                     & 0.6271                              & \textbf{0.6394}                     & 0.5932                              & 0.7109                              & \textbf{0.7342}                     & \textbf{0.6732}                     & 0.6562                              \\ \hline
        lstm\_3  & \textbf{0.6632}                     & 0.6598                              & 0.6276                              & \textbf{0.6624}                     & \textbf{0.7505}                     & 0.6083                              & \textbf{0.6836}                     & 0.6342                              \\ \hline
        lstm\_4  & 0.6598                              & \textbf{0.6644}                     & 0.6151                              & \textbf{0.6325}                     & \textbf{0.7972}                     & 0.7342                              & \textbf{\cellcolor{green!50}0.6944} & 0.6796                              \\ \hline
        lstm\_5  & \textbf{\cellcolor{green!50}0.6711} & 0.6587                              & \textbf{0.6347}                     & 0.6191                              & 0.7575                              & \textbf{0.7692}                     & \textbf{0.6907}                     & 0.6860                              \\ \hline
        lstm\_6  & \textbf{0.6677}                     & 0.6632                              & 0.6470                              & \textbf{0.6541}                     & \textbf{0.6923}                     & 0.6480                              & \textbf{0.6689}                     & 0.6510                              \\ \hline
        lstm\_7  & \textbf{0.6700}                     & 0.6531                              & \textbf{0.6395}                     & 0.6150                              & 0.7319                              & \textbf{0.7599}                     & \textbf{0.6826}                     & 0.6798                              \\ \hline
        lstm\_8  & 0.6429                              & \textbf{0.6474}                     & \textbf{0.6506}                     & 0.6291                              & 0.5687                              & \textbf{0.6643}                     & 0.6069                              & \textbf{0.6462}                     \\ \hline
        lstm\_9  & 0.6485                              & \textbf{0.6644}                     & 0.6006                              & \textbf{\cellcolor{green!50}0.6823} & \textbf{\cellcolor{green!50}0.8205} & 0.5757                              & \textbf{0.6935}                     & 0.6245                              \\ \hline
        lstm\_10 & \textbf{0.6644}                     & 0.6531                              & \textbf{0.6460}                     & 0.6150                              & 0.6806                              & \textbf{0.7599}                     & 0.6628                              & \textbf{0.6798}                     \\ \hline
        lstm\_11 & 0.6361                              & \textbf{0.6610}                     & \textbf{0.6389}                     & 0.6267                              & 0.5734                              & \textbf{0.7435}                     & 0.6044                              & \textbf{0.6801}                     \\ \hline
        lstm\_12 & 0.6689                              & \textbf{\cellcolor{green!50}0.6689} & \textbf{0.6497}                     & 0.6387                              & 0.6876                              & \textbf{0.7296}                     & 0.6681                              & \textbf{0.6811}                     \\ \hline
        lstm\_13 & \textbf{0.6519}                     & 0.6497                              & 0.6170                              & \textbf{0.6178}                     & \textbf{0.7435}                     & 0.7272                              & \textbf{0.6744}                     & 0.6680                              \\ \hline
        lstm\_14 & 0.6610                              & \textbf{0.6632}                     & \textbf{\cellcolor{green!50}0.6532} & 0.6367                              & 0.6410                              & \textbf{0.7109}                     & 0.6470                              & \textbf{0.6718}                     \\ \hline
        lstm\_15 & 0.6576                              & \textbf{0.6598}                     & 0.6211                              & \textbf{0.6385}                     & \textbf{0.7529}                     & 0.6876                              & \textbf{0.6807}                     & 0.6621                              \\ \hline
        lstm\_16 & 0.6677                              & \textbf{0.6677}                     & 0.6464                              & \textbf{0.6496}                     & \textbf{0.6946}                     & 0.6829                              & \textbf{0.6696}                     & 0.6659                              \\ \hline
        cnn\_1   & \textbf{0.6519}                     & 0.6485                              & \textbf{0.6416}                     & 0.6189                              & 0.6386                              & \textbf{0.7156}                     & 0.6401                              & \textbf{0.6637}                     \\ \hline
        cnn\_2   & 0.6316                              & \textbf{0.6485}                     & 0.6107                              & \textbf{0.6365}                     & \textbf{0.6620}                     & 0.6410                              & 0.6353                              & \textbf{0.6387}                     \\ \hline
        cnn\_3   & 0.6203                              & \textbf{0.6384}                     & 0.5954                              & \textbf{0.6326}                     & \textbf{0.6759}                     & 0.6060                              & \textbf{0.6331}                     & 0.6190                              \\ \hline
        cnn\_4   & 0.6440                              & \textbf{0.6508}                     & \textbf{0.6515}                     & 0.6345                              & 0.5710                              & \textbf{0.6596}                     & 0.6086                              & \textbf{0.6468}                     \\ \hline
        cnn\_5   & 0.6327                              & \textbf{0.6395}                     & 0.6125                              & \textbf{0.6200}                     & 0.6596                              & \textbf{0.6620}                     & 0.6352                              & \textbf{0.6403}                     \\ \hline
        cnn\_6   & 0.6395                              & \textbf{0.6598}                     & 0.6238                              & \textbf{0.6422}                     & 0.6456                              & \textbf{0.6736}                     & 0.6345                              & \textbf{0.6575}                     \\ \hline
        cnn\_7   & 0.6225                              & \textbf{0.6237}                     & \textbf{0.5975}                     & 0.5810                              & 0.6783                              & \textbf{\cellcolor{green!50}0.8018} & 0.6353                              & \textbf{0.6738}                     \\ \hline
        cnn\_8   & 0.6305                              & \textbf{0.6485}                     & 0.6153                              & \textbf{0.6347}                     & 0.6340                              & \textbf{0.6480}                     & 0.6245                              & \textbf{0.6412}                     \\ \hline
        cnn\_9   & \textbf{0.6406}                     & 0.6338                              & \textbf{0.6241}                     & 0.6143                              & 0.6503                              & \textbf{0.6573}                     & \textbf{0.6369}                     & 0.6351                              \\ \hline
        cnn\_10  & 0.6293                              & \textbf{0.6440}                     & 0.6150                              & \textbf{0.6350}                     & \textbf{0.6293}                     & 0.6247                              & 0.6221                              & \textbf{0.6298}                     \\ \hline
        cnn\_11  & \textbf{0.6395}                     & 0.6327                              & \textbf{0.6227}                     & 0.6111                              & 0.6503                              & \textbf{0.6666}                     & 0.6362                              & \textbf{0.6376}                     \\ \hline
    \end{tabular}
\end{table}

% \begin{tabular}{l|c|r}
%     \hline
%     Some & \textbf{\cellcolor{green!50}coloured} & \textbf{\cellcolor{green!50}TTT} \\
%     \hline
% \end{tabular}


\begin{table}[p] \centering
    \caption{Embedding FastText o długości 300, zbiór A}
    \label{tab:wyniki_fasttext_A}
    \begin{tabular}{|c|c|c|c|c|c|c|c|c|}    \hline

                 & \multicolumn{8}{c|}{metryki}                                                                                                                                                                                                                                                                                  \\ \hline
        X        & \multicolumn{2}{c|}{accuracy}       & \multicolumn{2}{c|}{precision}      & \multicolumn{2}{c|}{recall}         & \multicolumn{2}{c|}{f1}                                                                                                                                                                     \\ \hline
        sieć     & N                                   & P                                   & N                                   & P                                   & N                                   & P                                   & N                                   & P                                   \\ \hline
        dense\_1 & \textbf{0.6598}                     & 0.6497                              & \textbf{0.6355}                     & 0.6307                              & \textbf{0.6993}                     & 0.6689                              & \textbf{0.6659}                     & 0.6493                              \\ \hline
        dense\_2 & \textbf{0.6677}                     & 0.6576                              & \textbf{0.6490}                     & 0.6334                              & 0.6853                              & \textbf{0.6969}                     & \textbf{0.6666}                     & 0.6637                              \\ \hline
        dense\_3 & \textbf{0.6440}                     & 0.6327                              & \textbf{0.6126}                     & 0.6                                 & 0.7226                              & \textbf{0.7272}                     & \textbf{0.6631}                     & 0.6575                              \\ \hline
        dense\_4 & \textbf{0.6655}                     & 0.6418                              & \textbf{0.6501}                     & 0.6129                              & 0.6713                              & \textbf{0.7086}                     & \textbf{0.6605}                     & 0.6572                              \\ \hline
        lstm\_1  & 0.6813                              & \textbf{0.6824}                     & \textbf{\cellcolor{green!50}0.6909} & 0.6439                              & 0.6200                              & \textbf{0.7715}                     & 0.6535                              & \textbf{\cellcolor{green!50}0.7020} \\ \hline
        lstm\_2  & 0.6485                              & \textbf{0.6655}                     & 0.6322                              & \textbf{0.6405}                     & 0.6573                              & \textbf{0.7062}                     & 0.6445                              & \textbf{0.6718}                     \\ \hline
        lstm\_3  & 0.6632                              & \textbf{0.6813}                     & 0.6219                              & \textbf{0.6622}                     & \textbf{\cellcolor{green!50}0.7785} & 0.6993                              & \textbf{\cellcolor{green!50}0.6915} & 0.6802                              \\ \hline
        lstm\_4  & \textbf{0.6779}                     & 0.6576                              & \textbf{0.6463}                     & 0.6184                              & 0.7412                              & \textbf{0.7668}                     & \textbf{0.6905}                     & 0.6847                              \\ \hline
        lstm\_5  & \textbf{0.6870}                     & 0.6677                              & \textbf{0.6681}                     & 0.6537                              & \textbf{0.7039}                     & 0.6689                              & \textbf{0.6855}                     & 0.6612                              \\ \hline
        lstm\_6  & \textbf{\cellcolor{green!50}0.6926} & 0.6813                              & \textbf{0.6780}                     & 0.6496                              & 0.6969                              & \textbf{0.7435}                     & 0.6873                              & \textbf{0.6934}                     \\ \hline
        lstm\_7  & \textbf{0.6813}                     & 0.6553                              & \textbf{0.6689}                     & 0.6152                              & 0.6783                              & \textbf{0.7715}                     & 0.6736                              & \textbf{0.6845}                     \\ \hline
        lstm\_8  & \textbf{0.6711}                     & 0.6700                              & 0.6431                              & \textbf{0.6751}                     & \textbf{0.7226}                     & 0.6153                              & \textbf{0.6805}                     & 0.6439                              \\ \hline
        lstm\_9  & \textbf{0.6779}                     & 0.6497                              & \textbf{0.6698}                     & 0.6064                              & 0.6620                              & \textbf{\cellcolor{green!50}0.7902} & 0.6658                              & \textbf{0.6862}                     \\ \hline
        lstm\_10 & \textbf{0.6836}                     & 0.6474                              & \textbf{0.6769}                     & 0.6338                              & \textbf{0.6643}                     & 0.6456                              & \textbf{0.6705}                     & 0.6397                              \\ \hline
        lstm\_11 & \textbf{0.6542}                     & 0.6508                              & \textbf{0.6242}                     & 0.6162                              & 0.7202                              & \textbf{0.7412}                     & 0.6688                              & \textbf{0.6730}                     \\ \hline
        lstm\_12 & \textbf{0.6836}                     & 0.6768                              & \textbf{0.6712}                     & 0.6682                              & \textbf{0.6806}                     & 0.6620                              & \textbf{0.6759}                     & 0.6651                              \\ \hline
        lstm\_13 & \textbf{0.6779}                     & 0.6757                              & 0.6538                              & \textbf{0.6748}                     & \textbf{0.7132}                     & 0.6386                              & \textbf{0.6822}                     & 0.6562                              \\ \hline
        lstm\_14 & 0.6711                              & \textbf{\cellcolor{green!50}0.6937} & 0.6526                              & \textbf{\cellcolor{green!50}0.7219} & \textbf{0.6876}                     & 0.5990                              & \textbf{0.6696}                     & 0.6547                              \\ \hline
        lstm\_15 & \textbf{0.6790}                     & 0.6508                              & \textbf{0.6593}                     & 0.6102                              & 0.6993                              & \textbf{0.7738}                     & 0.6787                              & \textbf{0.6824}                     \\ \hline
        lstm\_16 & \textbf{0.6824}                     & 0.6757                              & 0.6674                              & \textbf{0.6961}                     & \textbf{0.6876}                     & 0.5874                              & \textbf{0.6773}                     & 0.6371                              \\ \hline
        cnn\_1   & \textbf{0.6564}                     & 0.6384                              & \textbf{0.6310}                     & 0.6142                              & \textbf{0.7016}                     & 0.6829                              & \textbf{0.6644}                     & 0.6467                              \\ \hline
        cnn\_2   & \textbf{0.6542}                     & 0.6429                              & \textbf{0.6413}                     & 0.6263                              & 0.6503                              & \textbf{0.6526}                     & \textbf{0.6458}                     & 0.6392                              \\ \hline
        cnn\_3   & 0.6271                              & \textbf{0.6429}                     & 0.6092                              & \textbf{0.6287}                     & 0.6433                              & \textbf{0.6433}                     & 0.6258                              & \textbf{0.6359}                     \\ \hline
        cnn\_4   & 0.6372                              & \textbf{0.6474}                     & \textbf{0.6310}                     & 0.6258                              & 0.6060                              & \textbf{0.6783}                     & 0.6183                              & \textbf{0.6510}                     \\ \hline
        cnn\_5   & \textbf{0.6598}                     & 0.6542                              & \textbf{0.6428}                     & 0.6382                              & \textbf{0.6713}                     & 0.6620                              & \textbf{0.6567}                     & 0.6498                              \\ \hline
        cnn\_6   & 0.6485                              & \textbf{0.6519}                     & 0.6213                              & \textbf{0.6257}                     & \textbf{0.7039}                     & 0.7016                              & 0.6601                              & \textbf{0.6615}                     \\ \hline
        cnn\_7   & 0.6644                              & \textbf{0.6644}                     & 0.6392                              & \textbf{0.6434}                     & \textbf{0.7062}                     & 0.6899                              & \textbf{0.6710}                     & 0.6659                              \\ \hline
        cnn\_8   & \textbf{0.6451}                     & 0.6395                              & \textbf{0.6220}                     & 0.6211                              & \textbf{0.6829}                     & 0.6573                              & \textbf{0.6511}                     & 0.6387                              \\ \hline
        cnn\_9   & 0.6372                              & \textbf{0.6485}                     & 0.6168                              & \textbf{0.6288}                     & 0.6643                              & \textbf{0.6713}                     & 0.6397                              & \textbf{0.6493}                     \\ \hline
        cnn\_10  & 0.6485                              & \textbf{0.6485}                     & 0.6277                              & \textbf{0.6299}                     & \textbf{0.6759}                     & 0.6666                              & \textbf{0.6509}                     & 0.6477                              \\ \hline
        cnn\_11  & 0.6451                              & \textbf{0.6542}                     & 0.6321                              & \textbf{0.6363}                     & 0.6410                              & \textbf{0.6689}                     & 0.6365                              & \textbf{0.6522}                     \\ \hline
    \end{tabular}
\end{table}



\begin{table}[p]  \centering
    \caption{Embedding ELMo o długości 1024, zbiór A}
    \label{tab:wyniki_elmo_A}
    \begin{tabular}{|c|c|c|c|c|c|c|c|c|}    \hline
                 & \multicolumn{8}{c|}{metryki}                                                                                                                                                                                                                                                                                  \\ \hline
        X        & \multicolumn{2}{c|}{accuracy}       & \multicolumn{2}{c|}{precision}      & \multicolumn{2}{c|}{recall}         & \multicolumn{2}{c|}{f1}                                                                                                                                                                     \\ \hline
        sieć     & N                                   & P                                   & N                                   & P                                   & N                                   & P                                   & N                                   & P                                   \\ \hline
        dense\_1 & 0.6531                              & \textbf{0.6779}                     & 0.6392                              & \textbf{0.6487}                     & 0.6526                              & \textbf{0.7319}                     & 0.6459                              & \textbf{0.6878}                     \\ \hline
        dense\_2 & \textbf{0.6519}                     & 0.6429                              & \textbf{0.6273}                     & 0.6184                              & \textbf{0.6946}                     & 0.6876                              & \textbf{0.6592}                     & 0.6512                              \\ \hline
        dense\_3 & 0.6203                              & \textbf{0.6418}                     & 0.5882                              & \textbf{0.6261}                     & \textbf{0.7226}                     & 0.6480                              & \textbf{0.6485}                     & 0.6368                              \\ \hline
        dense\_4 & 0.6406                              & \textbf{0.6485}                     & 0.6065                              & \textbf{0.6520}                     & \textbf{0.7365}                     & 0.5897                              & \textbf{0.6652}                     & 0.6193                              \\ \hline
        lstm\_1  & \textbf{0.6836}                     & 0.6655                              & \textbf{0.6795}                     & 0.6164                              & 0.6573                              & \textbf{\cellcolor{green!50}0.8205} & 0.6682                              & \textbf{0.704}                      \\ \hline
        lstm\_2  & \textbf{0.7152}                     & 0.6903                              & \textbf{0.7015}                     & 0.6611                              & 0.7179                              & \textbf{0.7412}                     & \textbf{0.7096}                     & 0.6989                              \\ \hline
        lstm\_3  & \textbf{0.6926}                     & 0.6711                              & \textbf{0.6891}                     & 0.6443                              & 0.6666                              & \textbf{0.7179}                     & 0.6777                              & \textbf{0.6791}                     \\ \hline
        lstm\_4  & 0.6802                              & \textbf{0.6892}                     & \textbf{0.6666}                     & 0.6492                              & 0.6806                              & \textbf{0.7808}                     & 0.6735                              & \textbf{0.7089}                     \\ \hline
        lstm\_5  & \textbf{0.6971}                     & 0.6768                              & \textbf{0.6687}                     & 0.6382                              & 0.7435                              & \textbf{0.7692}                     & \textbf{0.7041}                     & 0.6976                              \\ \hline
        lstm\_6  & 0.6926                              & \textbf{\cellcolor{green!50}0.7129} & 0.6772                              & \textbf{0.6826}                     & 0.6993                              & \textbf{0.7622}                     & 0.6880                              & \textbf{\cellcolor{green!50}0.7202} \\ \hline
        lstm\_7  & \textbf{\cellcolor{green!50}0.7186} & 0.7118                              & \textbf{0.7}                        & 0.6986                              & \textbf{0.7342}                     & 0.7132                              & \textbf{\cellcolor{green!50}0.7167} & 0.7058                              \\ \hline
        lstm\_8  & 0.6734                              & \textbf{0.6926}                     & 0.6933                              & \textbf{\cellcolor{green!50}0.7524} & \textbf{0.5850}                     & 0.5454                              & \textbf{0.6346}                     & 0.6324                              \\ \hline
        lstm\_9  & 0.6903                              & \textbf{0.7050}                     & 0.6962                              & \textbf{0.7038}                     & 0.6410                              & \textbf{0.6759}                     & 0.6674                              & \textbf{0.6896}                     \\ \hline
        lstm\_10 & \textbf{0.6892}                     & 0.6847                              & \textbf{0.6824}                     & 0.6820                              & \textbf{0.6713}                     & 0.6550                              & \textbf{0.6768}                     & 0.6682                              \\ \hline
        lstm\_11 & 0.6757                              & \textbf{0.6802}                     & \textbf{0.6658}                     & 0.6442                              & 0.6643                              & \textbf{0.7599}                     & 0.6651                              & \textbf{0.6973}                     \\ \hline
        lstm\_12 & \textbf{0.6971}                     & 0.6813                              & 0.6716                              & \textbf{0.6981}                     & \textbf{0.7342}                     & 0.6037                              & \textbf{0.7015}                     & 0.6475                              \\ \hline
        lstm\_13 & \textbf{0.7039}                     & 0.6711                              & \textbf{0.6937}                     & 0.6699                              & \textbf{0.6969}                     & 0.6340                              & \textbf{0.6953}                     & 0.6514                              \\ \hline
        lstm\_14 & \textbf{0.6960}                     & 0.6903                              & 0.66                                & \textbf{0.7066}                     & \textbf{\cellcolor{green!50}0.7692} & 0.6177                              & \textbf{0.7104}                     & 0.6592                              \\ \hline
        lstm\_15 & \textbf{0.6949}                     & 0.6892                              & 0.6593                              & \textbf{0.6816}                     & \textbf{0.7668}                     & 0.6736                              & \textbf{0.7090}                     & 0.6776                              \\ \hline
        lstm\_16 & 0.6937                              & \textbf{0.7016}                     & \textbf{\cellcolor{green!50}0.7078} & 0.6813                              & 0.6270                              & \textbf{0.7226}                     & 0.6650                              & \textbf{0.7013}                     \\ \hline
        cnn\_1   & 0.6282                              & \textbf{0.6418}                     & 0.6086                              & \textbf{0.6296}                     & \textbf{0.6526}                     & 0.6340                              & 0.6299                              & \textbf{0.6318}                     \\ \hline
        cnn\_2   & 0.6508                              & \textbf{0.6531}                     & \textbf{0.6333}                     & 0.6326                              & 0.6643                              & \textbf{0.6783}                     & 0.6484                              & \textbf{0.6546}                     \\ \hline
        cnn\_3   & \textbf{0.6497}                     & 0.6485                              & \textbf{0.64}                       & 0.6340                              & 0.6340                              & \textbf{0.6503}                     & 0.6370                              & \textbf{0.6421}                     \\ \hline
        cnn\_4   & \textbf{0.6564}                     & 0.6542                              & \textbf{0.6443}                     & 0.6394                              & 0.6503                              & \textbf{0.6573}                     & 0.6473                              & \textbf{0.6482}                     \\ \hline
        cnn\_5   & \textbf{0.6598}                     & 0.6587                              & \textbf{0.6454}                     & 0.6420                              & 0.6620                              & \textbf{0.6689}                     & 0.6536                              & \textbf{0.6552}                     \\ \hline
        cnn\_6   & \textbf{0.6519}                     & 0.6203                              & \textbf{0.6437}                     & 0.6120                              & \textbf{0.6317}                     & 0.5920                              & \textbf{0.6376}                     & 0.6018                              \\ \hline
        cnn\_7   & 0.6610                              & \textbf{0.6632}                     & \textbf{0.6592}                     & 0.6356                              & 0.6223                              & \textbf{0.7156}                     & 0.6402                              & \textbf{0.6732}                     \\ \hline
        cnn\_8   & \textbf{0.6451}                     & 0.6406                              & \textbf{0.6258}                     & 0.6158                              & 0.6666                              & \textbf{0.6876}                     & 0.6455                              & \textbf{0.6497}                     \\ \hline
        cnn\_9   & \textbf{0.6519}                     & 0.6384                              & 0.6318                              & \textbf{0.6339}                     & \textbf{0.6759}                     & 0.6013                              & \textbf{0.6531}                     & 0.6172                              \\ \hline
        cnn\_10  & \textbf{0.6632}                     & 0.6497                              & 0.6427                              & \textbf{0.6461}                     & \textbf{0.6876}                     & 0.6130                              & \textbf{0.6644}                     & 0.6291                              \\ \hline
        cnn\_11  & 0.6293                              & \textbf{0.6451}                     & 0.5980                              & \textbf{0.6125}                     & 0.7179                              & \textbf{0.7296}                     & 0.6525                              & \textbf{0.6659}                     \\ \hline
    \end{tabular}
\end{table}


\newpage % Rozdziały zaczynamy od nowej strony.
\subsubsection{Wyniki dla eksperymentu 1 - zbiór B}


\begin{table}[h] \centering
    \caption{Embedding Glove o długości 25, zbiór B}
    \label{tab:wyniki_glove_B}
    \begin{tabular} {|c|c|c|c|c|c|c|c|c| }    \hline
                 & \multicolumn{8}{c|}{metryki}                                                                                                                                                                                                                                                                             \\ \hline
        X        & \multicolumn{2}{c|}{accuracy}       & \multicolumn{2}{c|}{precision}      & \multicolumn{2}{c|}{recall}         & \multicolumn{2}{c|}{f1}                                                                                                                                                                \\ \hline
        sieć     & N                                   & P                                   & N                                   & P                              & N                                   & P                                   & N                                   & P                                   \\ \hline
        dense\_1 & 0.8477                              & \textbf{0.8593}                     & 0.8460                              & \textbf{0.8503}                & 0.8526                              & \textbf{0.8744}                     & 0.8493                              & \textbf{0.8622}                     \\ \hline
        dense\_2 & 0.8483                              & \textbf{0.8613}                     & 0.8349                              & \textbf{0.8551}                & 0.8709                              & \textbf{0.8724}                     & 0.8525                              & \textbf{0.8636}                     \\ \hline
        dense\_3 & 0.8433                              & \textbf{0.8604}                     & 0.8403                              & \textbf{0.8510}                & 0.8504                              & \textbf{0.8762}                     & 0.8453                              & \textbf{0.8634}                     \\ \hline
        dense\_4 & 0.8493                              & \textbf{0.8646}                     & 0.8455                              & \textbf{0.8731}                & \textbf{0.8574}                     & 0.8554                              & 0.8514                              & \textbf{0.8641}                     \\ \hline
        lstm\_1  & 0.9914                              & \textbf{\cellcolor{green!50}0.9921} & 0.9992                              & \textbf{0.9997}                & 0.9837                              & \textbf{0.9847}                     & 0.9914                              & \textbf{\cellcolor{green!50}0.9921} \\ \hline
        lstm\_2  & \textbf{0.9910}                     & 0.9909                              & 0.9984                              & \textbf{0.9989}                & \textbf{0.9837}                     & 0.9829                              & \textbf{0.9910}                     & 0.9909                              \\ \hline
        lstm\_3  & 0.9899                              & \textbf{0.9910}                     & 0.9966                              & \textbf{0.9987}                & 0.9832                              & \textbf{0.9834}                     & 0.9899                              & \textbf{0.9910}                     \\ \hline
        lstm\_4  & \textbf{0.9910}                     & 0.9906                              & 0.9977                              & \textbf{0.9982}                & \textbf{0.9844}                     & 0.9832                              & \textbf{0.9910}                     & 0.9906                              \\ \hline
        lstm\_5  & 0.9906                              & \textbf{0.9906}                     & \textbf{0.9989}                     & 0.9974                         & 0.9824                              & \textbf{0.9839}                     & 0.9906                              & \textbf{0.9906}                     \\ \hline
        lstm\_6  & \textbf{0.9911}                     & 0.9895                              & \textbf{0.9972}                     & 0.9969                         & \textbf{0.9852}                     & 0.9822                              & \textbf{0.9911}                     & 0.9895                              \\ \hline
        lstm\_7  & \textbf{0.9912}                     & 0.9909                              & 0.9987                              & \textbf{0.9992}                & \textbf{0.9839}                     & 0.9827                              & \textbf{0.9912}                     & 0.9909                              \\ \hline
        lstm\_8  & 0.9918                              & \textbf{0.9918}                     & \textbf{0.9989}                     & 0.9987                         & 0.9847                              & \textbf{0.9849}                     & 0.9917                              & \textbf{0.9918}                     \\ \hline
        lstm\_9  & \textbf{0.9915}                     & 0.9914                              & 0.9974                              & \textbf{\cellcolor{green!50}1} & \textbf{\cellcolor{green!50}0.9857} & 0.9829                              & \textbf{0.9915}                     & 0.9914                              \\ \hline
        lstm\_10 & \textbf{\cellcolor{green!50}0.9925} & 0.9918                              & 0.9994                              & \textbf{0.9997}                & \textbf{\cellcolor{green!50}0.9857} & 0.9839                              & \textbf{\cellcolor{green!50}0.9925} & 0.9917                              \\ \hline
        lstm\_11 & \textbf{0.9918}                     & 0.9914                              & 0.9982                              & \textbf{0.9982}                & \textbf{0.9854}                     & 0.9847                              & \textbf{0.9918}                     & 0.9914                              \\ \hline
        lstm\_12 & \textbf{0.9923}                     & 0.9916                              & \textbf{0.9994}                     & 0.9977                         & 0.9852                              & \textbf{\cellcolor{green!50}0.9857} & \textbf{0.9923}                     & 0.9916                              \\ \hline
        lstm\_13 & \textbf{0.9924}                     & 0.9916                              & \textbf{\cellcolor{green!50}0.9997} & 0.9992                         & \textbf{0.9852}                     & 0.9842                              & \textbf{0.9924}                     & 0.9916                              \\ \hline
        lstm\_14 & \textbf{0.9916}                     & 0.9911                              & 0.9994                              & \textbf{0.9994}                & \textbf{0.9839}                     & 0.9829                              & \textbf{0.9916}                     & 0.9911                              \\ \hline
        lstm\_15 & 0.9902                              & \textbf{0.9909}                     & 0.9974                              & \textbf{0.9989}                & \textbf{0.9832}                     & 0.9829                              & 0.9902                              & \textbf{0.9909}                     \\ \hline
        lstm\_16 & \textbf{0.9923}                     & 0.9907                              & \textbf{\cellcolor{green!50}0.9997} & 0.9984                         & \textbf{0.9849}                     & 0.9832                              & \textbf{0.9923}                     & 0.9907                              \\ \hline
        cnn\_1   & 0.9621                              & \textbf{0.9680}                     & 0.9723                              & \textbf{0.9814}                & 0.9518                              & \textbf{0.9546}                     & 0.9620                              & \textbf{0.9678}                     \\ \hline
        cnn\_2   & 0.9704                              & \textbf{0.9706}                     & 0.9847                              & \textbf{0.9850}                & 0.9561                              & \textbf{0.9561}                     & 0.9702                              & \textbf{0.9703}                     \\ \hline
        cnn\_3   & 0.9452                              & \textbf{0.9471}                     & 0.9630                              & \textbf{0.9704}                & \textbf{0.9268}                     & 0.9230                              & 0.9445                              & \textbf{0.9461}                     \\ \hline
        cnn\_4   & 0.9501                              & \textbf{0.9538}                     & 0.9701                              & \textbf{0.9733}                & 0.9295                              & \textbf{0.9338}                     & 0.9494                              & \textbf{0.9531}                     \\ \hline
        cnn\_5   & \textbf{0.9569}                     & 0.9563                              & \textbf{0.9833}                     & 0.9732                         & 0.9303                              & \textbf{0.9391}                     & \textbf{0.9560}                     & 0.9558                              \\ \hline
        cnn\_6   & \textbf{0.9919}                     & 0.9918                              & \textbf{0.9994}                     & 0.9984                         & 0.9844                              & \textbf{0.9852}                     & \textbf{0.9919}                     & 0.9918                              \\ \hline
        cnn\_7   & \textbf{0.9924}                     & \cellcolor{green!50}0.9921          & \textbf{\cellcolor{green!50}0.9997} & 0.9992                         & 0.9852                              & \textbf{0.9852}                     & \textbf{0.9924}                     & \cellcolor{green!50}0.9921          \\ \hline
        cnn\_8   & 0.9603                              & \textbf{0.9656}                     & 0.9796                              & \textbf{0.9796}                & 0.9408                              & \textbf{0.9516}                     & 0.9598                              & \textbf{0.9654}                     \\ \hline
        cnn\_9   & 0.9552                              & \textbf{0.9612}                     & 0.9722                              & \textbf{0.9809}                & 0.9378                              & \textbf{0.9413}                     & 0.9547                              & \textbf{0.9607}                     \\ \hline
        cnn\_10  & 0.9547                              & \textbf{0.9562}                     & 0.9739                              & \textbf{0.9742}                & 0.9351                              & \textbf{0.9378}                     & 0.9541                              & \textbf{0.9557}                     \\ \hline
        cnn\_11  & 0.9688                              & \textbf{0.9759}                     & 0.9842                              & \textbf{0.9909}                & 0.9533                              & \textbf{0.9609}                     & 0.9685                              & \textbf{0.9757}                     \\ \hline
    \end{tabular}
\end{table}



\begin{table}[p]  \centering
    \caption{Embedding FastText o długości 300, zbiór B}
    \label{tab:wyniki_fasttext_B}
    \begin{tabular} {|c|c|c|c|c|c|c|c|c| }    \hline
                 & \multicolumn{8}{c|}{metryki}                                                                                                                                                                                                                                                \\ \hline
        X        & \multicolumn{2}{c|}{accuracy} & \multicolumn{2}{c|}{precision}      & \multicolumn{2}{c|}{recall}    & \multicolumn{2}{c|}{f1}                                                                                                                                              \\ \hline
        sieć     & N                             & P                                   & N                              & P                              & N                          & P                                   & N                          & P                                   \\ \hline
        dense\_1 & 0.9297                        & \textbf{0.9303}                     & 0.9294                         & \textbf{0.9299}                & 0.9310                     & \textbf{0.9318}                     & 0.9302                     & \textbf{0.9309}                     \\ \hline
        dense\_2 & 0.9286                        & \textbf{0.9322}                     & 0.9221                         & \textbf{0.9341}                & \textbf{0.9373}            & 0.9310                              & 0.9296                     & \textbf{0.9326}                     \\ \hline
        dense\_3 & \textbf{0.9297}               & 0.9277                              & \textbf{0.9346}                & 0.9132                         & 0.9250                     & \textbf{0.9463}                     & \textbf{0.9298}            & 0.9294                              \\ \hline
        dense\_4 & 0.9293                        & \textbf{0.9310}                     & 0.9236                         & \textbf{0.9281}                & \textbf{0.9371}            & 0.9353                              & 0.9303                     & \textbf{0.9317}                     \\ \hline
        lstm\_1  & 0.9858                        & \textbf{0.9918}                     & 0.9948                         & \textbf{0.9997}                & 0.9769                     & \textbf{0.9839}                     & 0.9858                     & \textbf{0.9917}                     \\ \hline
        lstm\_2  & \textbf{0.9918}               & 0.9909                              & \textbf{0.9994}                & 0.9972                         & 0.9842                     & \textbf{0.9847}                     & \textbf{0.9917}            & 0.9909                              \\ \hline
        lstm\_3  & \textbf{0.9920}               & 0.9904                              & \textbf{0.9992}                & 0.9967                         & \textbf{0.9849}            & 0.9842                              & \textbf{0.9920}            & 0.9904                              \\ \hline
        lstm\_4  & 0.9911                        & \textbf{0.9911}                     & 0.9979                         & \textbf{0.9994}                & \textbf{0.9844}            & 0.9829                              & 0.9911                     & \textbf{0.9911}                     \\ \hline
        lstm\_5  & \textbf{0.9916}               & 0.9911                              & \textbf{0.9987}                & 0.9969                         & 0.9847                     & \textbf{0.9854}                     & \textbf{0.9916}            & 0.9911                              \\ \hline
        lstm\_6  & 0.9905                        & \textbf{0.9909}                     & 0.9964                         & \textbf{0.9984}                & \textbf{0.9847}            & 0.9834                              & 0.9905                     & \textbf{0.9909}                     \\ \hline
        lstm\_7  & \textbf{0.9921}               & 0.9900                              & 0.9984                         & \textbf{\cellcolor{green!50}1} & \textbf{0.9859}            & 0.9802                              & \textbf{0.9921}            & 0.9900                              \\ \hline
        lstm\_8  & 0.9920                        & \textbf{0.9921}                     & \textbf{\cellcolor{green!50}1} & 0.9992                         & 0.9842                     & \textbf{0.9852}                     & 0.9920                     & \textbf{0.9921}                     \\ \hline
        lstm\_9  & \textbf{0.9925}               & 0.9914                              & \textbf{\cellcolor{green!50}1} & 0.9969                         & 0.9852                     & \textbf{0.9859}                     & \textbf{0.9925}            & 0.9914                              \\ \hline
        lstm\_10 & \textbf{0.9923}               & 0.9919                              & 0.9989                         & \textbf{0.9997}                & \textbf{0.9857}            & 0.9842                              & \textbf{0.9923}            & 0.9919                              \\ \hline
        lstm\_11 & \textbf{0.9924}               & 0.9923                              & 0.9994                         & \textbf{\cellcolor{green!50}1} & \textbf{0.9854}            & 0.9847                              & \textbf{0.9924}            & 0.9922                              \\ \hline
        lstm\_12 & \textbf{0.9921}               & 0.9919                              & 0.9987                         & \textbf{0.9994}                & \textbf{0.9857}            & 0.9844                              & \textbf{0.9921}            & 0.9919                              \\ \hline
        lstm\_13 & 0.9915                        & \textbf{0.9921}                     & 0.9982                         & \textbf{0.9997}                & \textbf{0.9849}            & 0.9847                              & 0.9915                     & \textbf{0.9921}                     \\ \hline
        lstm\_14 & 0.9915                        & \textbf{0.9924}                     & 0.9967                         & \textbf{0.9982}                & \cellcolor{green!50}0.9864 & \textbf{\cellcolor{green!50}0.9867} & 0.9915                     & \textbf{0.9924}                     \\ \hline
        lstm\_15 & 0.9925                        & \textbf{0.9928}                     & 0.9992                         & \textbf{0.9994}                & 0.9859                     & \textbf{0.9862}                     & 0.9925                     & \textbf{0.9928}                     \\ \hline
        lstm\_16 & 0.9921                        & \textbf{0.9921}                     & 0.9989                         & \textbf{0.9994}                & \textbf{0.9854}            & 0.9849                              & 0.9921                     & \textbf{0.9921}                     \\ \hline
        cnn\_1   & 0.9901                        & \textbf{0.9901}                     & \textbf{0.9977}                & 0.9964                         & 0.9827                     & \textbf{0.9839}                     & 0.9901                     & \textbf{0.9901}                     \\ \hline
        cnn\_2   & 0.9902                        & \textbf{0.9914}                     & 0.9979                         & \textbf{0.9992}                & 0.9827                     & \textbf{0.9837}                     & 0.9902                     & \textbf{0.9914}                     \\ \hline
        cnn\_3   & \textbf{0.9875}               & 0.9866                              & \textbf{0.9964}                & 0.9956                         & \textbf{0.9787}            & 0.9776                              & \textbf{0.9874}            & 0.9865                              \\ \hline
        cnn\_4   & 0.9866                        & \textbf{0.9877}                     & \textbf{0.9959}                & 0.9951                         & 0.9774                     & \textbf{0.9804}                     & 0.9865                     & \textbf{0.9877}                     \\ \hline
        cnn\_5   & \textbf{0.9896}               & 0.9889                              & \textbf{0.9977}                & 0.9974                         & \textbf{0.9817}            & 0.9804                              & \textbf{0.9896}            & 0.9888                              \\ \hline
        cnn\_6   & \cellcolor{green!50}0.9929    & \textbf{\cellcolor{green!50}0.9929} & \cellcolor{green!50}1          & \textbf{\cellcolor{green!50}1} & 0.9859                     & \textbf{0.9859}                     & \cellcolor{green!50}0.9929 & \textbf{\cellcolor{green!50}0.9929} \\ \hline
        cnn\_7   & 0.9926                        & \textbf{0.9926}                     & 0.9997                         & \textbf{0.9997}                & 0.9857                     & \textbf{0.9857}                     & 0.9926                     & \textbf{0.9926}                     \\ \hline
        cnn\_8   & 0.9910                        & \textbf{0.9911}                     & \textbf{0.9992}                & 0.9979                         & 0.9829                     & \textbf{0.9844}                     & 0.9910                     & \textbf{0.9911}                     \\ \hline
        cnn\_9   & 0.9897                        & \textbf{0.9900}                     & \textbf{0.9984}                & 0.9982                         & 0.9812                     & \textbf{0.9819}                     & 0.9897                     & \textbf{0.9900}                     \\ \hline
        cnn\_10  & \textbf{0.9905}               & 0.9886                              & \textbf{0.9979}                & 0.9961                         & \textbf{0.9832}            & 0.9812                              & \textbf{0.9905}            & 0.9886                              \\ \hline
        cnn\_11  & \textbf{0.9918}               & 0.9916                              & \textbf{0.9992}                & 0.9989                         & 0.9844                     & \textbf{0.9844}                     & \textbf{0.9917}            & 0.9916                              \\ \hline
    \end{tabular}
\end{table}



\begin{table}[p] \centering
    \caption{Embedding ELMo o długości 1024, zbiór B}
    \label{tab:wyniki_elmo_B}
    \begin{tabular} {|c|c|c|c|c|c|c|c|c| }    \hline
                 & \multicolumn{8}{c|}{metryki}                                                                                                                                                                                                                                                            \\ \hline
        X        & \multicolumn{2}{c|}{accuracy}       & \multicolumn{2}{c|}{precision}      & \multicolumn{2}{c|}{recall} & \multicolumn{2}{c|}{f1}                                                                                                                                                       \\ \hline
        sieć     & N                                   & P                                   & N                           & P                                   & N                           & P                                    & N                                    & P                           \\ \hline
        dense\_1 & 0.9298                              & \textbf{0.9304}                     & 0.9284                      & \textbf{0.9298}                     & 0.93114                     & \textbf{0.93198}                     & 0.93045                              & \textbf{0.93108}            \\ \hline
        dense\_2 & 0.9294                              & \textbf{0.9323}                     & 0.9220                      & \textbf{0.9337}                     & \textbf{0.93759}            & 0.93179                              & 0.93017                              & \textbf{0.93298}            \\ \hline
        dense\_3 & \textbf{0.9301}                     & 0.9282                              & \textbf{0.9340}             & 0.9126                              & 0.9255                      & \textbf{0.94666}                     & \textbf{0.92994}                     & 0.92959                     \\ \hline
        dense\_4 & 0.9297                              & \textbf{0.9319}                     & 0.9234                      & \textbf{0.9273}                     & \textbf{0.93794}            & 0.93532                              & 0.9307                               & \textbf{0.93254}            \\ \hline
        lstm\_1  & 0.9860                              & \textbf{0.9922}                     & 0.9940                      & \textbf{0.9994}                     & 0.97705                     & \textbf{0.98439}                     & 0.98642                              & \textbf{0.99252}            \\ \hline
        lstm\_2  & \textbf{0.9920}                     & 0.9912                              & \textbf{0.9987}             & 0.9966                              & 0.9844                      & \textbf{0.98543}                     & \textbf{0.992}                       & 0.99098                     \\ \hline
        lstm\_3  & \textbf{0.9927}                     & 0.9911                              & \textbf{0.9986}             & 0.9959                              & \textbf{0.98559}            & 0.98435                              & \textbf{0.99286}                     & 0.99065                     \\ \hline
        lstm\_4  & \textbf{0.9920}                     & 0.9914                              & 0.9975                      & \textbf{0.9984}                     & \textbf{0.98524}            & 0.98304                              & \textbf{0.99189}                     & 0.9914                      \\ \hline
        lstm\_5  & \textbf{0.9919}                     & 0.9916                              & \textbf{0.9978}             & 0.9967                              & 0.98482                     & \textbf{0.98605}                     & \textbf{0.99208}                     & 0.99192                     \\ \hline
        lstm\_6  & \textbf{0.9913}                     & 0.9912                              & 0.9961                      & \textbf{0.9974}                     & \textbf{0.985}              & 0.98406                              & 0.99134                              & \textbf{0.99184}            \\ \hline
        lstm\_7  & \textbf{0.9921}                     & 0.9900                              & 0.9981                      & \textbf{\cellcolor{green!50}0.9998} & \textbf{0.98593}            & 0.98057                              & \textbf{0.99305}                     & 0.99028                     \\ \hline
        lstm\_8  & 0.9925                              & \textbf{0.9927}                     & 0.9990                      & \textbf{0.9991}                     & 0.98485                     & \textbf{0.98581}                     & 0.99246                              & \textbf{0.9931}             \\ \hline
        lstm\_9  & \textbf{\cellcolor{green!50}0.9934} & 0.9921                              & \textbf{0.9990}             & 0.9964                              & 0.98605                     & \textbf{0.98606}                     & \textbf{0.99289}                     & 0.99152                     \\ \hline
        lstm\_10 & \textbf{0.9929}                     & 0.9926                              & 0.9984                      & \textbf{0.9990}                     & \textbf{0.9865}             & 0.98472                              & 0.99235                              & \textbf{0.99236}            \\ \hline
        lstm\_11 & \textbf{0.9928}                     & 0.9924                              & 0.9985                      & \textbf{0.9990}                     & \textbf{0.98632}            & 0.98498                              & 0.99256                              & \textbf{0.99282}            \\ \hline
        lstm\_12 & \textbf{0.9928}                     & 0.9927                              & 0.9977                      & \textbf{0.9986}                     & \textbf{0.98639}            & 0.98498                              & \textbf{0.99249}                     & 0.99244                     \\ \hline
        lstm\_13 & 0.9918                              & \textbf{0.9928}                     & 0.9972                      & \textbf{0.9996}                     & \textbf{0.98567}            & 0.98498                              & 0.99171                              & \textbf{0.993}              \\ \hline
        lstm\_14 & 0.9923                              & \textbf{0.9929}                     & 0.9958                      & \textbf{0.9981}                     & \cellcolor{green!50}0.98688 & \textbf{\cellcolor{green!50}0.98766} & 0.99215                              & \textbf{0.99283}            \\ \hline
        lstm\_15 & 0.9927                              & \textbf{\cellcolor{green!50}0.9937} & 0.9989                      & \textbf{0.9992}                     & \textbf{0.98662}            & 0.98655                              & \textbf{0.99304}                     & 0.99301                     \\ \hline
        lstm\_16 & 0.9923                              & \textbf{0.9930}                     & 0.9985                      & \textbf{0.9986}                     & \textbf{0.98604}            & 0.98511                              & \textbf{0.9928}                      & 0.99251                     \\ \hline
        cnn\_1   & 0.9904                              & \textbf{0.9906}                     & \textbf{0.9968}             & 0.9955                              & 0.98328                     & \textbf{0.98399}                     & 0.99016                              & \textbf{0.99081}            \\ \hline
        cnn\_2   & 0.9909                              & \textbf{0.9916}                     & 0.9976                      & \textbf{0.9982}                     & 0.98331                     & \textbf{0.9839}                      & 0.99052                              & \textbf{0.9923}             \\ \hline
        cnn\_3   & \textbf{0.9875}                     & 0.9872                              & \textbf{0.9957}             & 0.9946                              & \textbf{0.97881}            & 0.97797                              & \textbf{0.98777}                     & 0.98652                     \\ \hline
        cnn\_4   & 0.9872                              & \textbf{0.9885}                     & \textbf{0.9952}             & 0.9950                              & 0.97744                     & \textbf{0.98136}                     & 0.98727                              & \textbf{0.98791}            \\ \hline
        cnn\_5   & \textbf{0.9905}                     & 0.9890                              & 0.9970                      & \textbf{0.9971}                     & \textbf{0.9823}             & 0.98082                              & \textbf{0.9906}                      & 0.98882                     \\ \hline
        cnn\_6   & 0.9929                              & \textbf{0.9934}                     & \cellcolor{green!50}0.9994  & \textbf{0.9996}                     & \textbf{0.98672}            & 0.98624                              & \textbf{\cellcolor{green!50}0.99376} & \cellcolor{green!50}0.99332 \\ \hline
        cnn\_7   & 0.9927                              & \textbf{0.9931}                     & \textbf{0.9992}             & 0.9988                              & \textbf{0.98617}            & 0.98586                              & 0.99301                              & \textbf{0.99305}            \\ \hline
        cnn\_8   & 0.9916                              & \textbf{0.9916}                     & \textbf{0.9984}             & 0.997                               & 0.98312                     & \textbf{0.98467}                     & 0.99141                              & \textbf{0.99189}            \\ \hline
        cnn\_9   & 0.9899                              & \textbf{0.9905}                     & \textbf{0.9977}             & 0.9976                              & 0.98182                     & \textbf{0.9826}                      & 0.99001                              & \textbf{0.9904}             \\ \hline
        cnn\_10  & \textbf{0.9909}                     & 0.9888                              & \textbf{0.9971}             & 0.9957                              & \textbf{0.98334}            & 0.98162                              & \textbf{0.99067}                     & 0.98889                     \\ \hline
        cnn\_11  & 0.9924                              & \textbf{0.9925}                     & 0.9983                      & \textbf{0.9986}                     & \textbf{0.98536}            & 0.98486                              & 0.99175                              & \textbf{0.99182}            \\ \hline
    \end{tabular}
\end{table}


%%%%%%%%%%%%%%%%%%%%%%%%%%%%%%%%%%%%%%%%%%%%%%%%%%%%%%
\newpage % Rozdziały zaczynamy od nowej strony.
\subsubsection{Wyniki dla eksperymentu 2}


%todo:
\colorbox{yellow}{todo:}\\
//tablea gdzie jest jakość na poziomie 50-55 procent



%todo:
\colorbox{yellow}{todo:}\\
//tabela dla roznych embedingow pokazujaca, że wartwa konwolucyjna jest o 5 procent gorsze od LSTM

%%%%%%%%%%%%%%%%%%%%%%%%%%%%%%%%%%%%%%%%%%%%%%%%%%%%%%
\newpage % Rozdziały zaczynamy od nowej strony.
\subsubsection{Wyniki dla eksperymentu 3}


\begin{table}[h] \label{tab:wyniki_glove_B_lim} \centering
    \caption{Embedding Glove o długości 25, zbiór B, ograniczony do równolicznych klas po 2000 rekordów}
    \begin{tabular} {|c|c|c|c|c|c|c|c|c| }    \hline
                 & \multicolumn{4}{c|}{metryki}                               \\ \hline
        sieć     & accuracy                     & precision & recall & f1     \\ \hline
        dense\_1 & 0.8275                       & 0.7985    & 0.8492 & 0.8230 \\ \hline
        dense\_2 & 0.8275                       & 0.7970    & 0.8518 & 0.8235 \\ \hline
        dense\_3 & 0.8175                       & 0.7944    & 0.8280 & 0.8108 \\ \hline
        dense\_4 & 0.8325                       & 0.7890    & 0.8809 & 0.8325 \\ \hline
        lstm\_1  & 0.8937                       & 0.8727    & 0.9074 & 0.8897 \\ \hline
        lstm\_2  & 0.9337                       & 0.9452    & 0.9126 & 0.9286 \\ \hline
        lstm\_3  & 0.9412                       & 0.9460    & 0.9285 & 0.9372 \\ \hline
        lstm\_4  & 0.9375                       & 0.9852    & 0.8809 & 0.9301 \\ \hline
        lstm\_5  & 0.9437                       & 0.9561    & 0.9232 & 0.9394 \\ \hline
        lstm\_6  & 0.9287                       & 0.9302    & 0.9179 & 0.9241 \\ \hline
        lstm\_7  & 0.8975                       & 0.9       & 0.8809 & 0.8903 \\ \hline
        lstm\_8  & 0.9475                       & 0.9516    & 0.9365 & 0.944  \\ \hline
        lstm\_9  & 0.9625                       & 0.9578    & 0.9629 & 0.9604 \\ \hline
        lstm\_10 & 0.9675                       & 0.9607    & 0.9708 & 0.9657 \\ \hline
        lstm\_11 & 0.9525                       & 0.9336    & 0.9682 & 0.9506 \\ \hline
        lstm\_12 & 0.95                         & 0.9642    & 0.9285 & 0.9460 \\ \hline
        lstm\_13 & 0.9662                       & 0.9705    & 0.9576 & 0.9640 \\ \hline
        lstm\_14 & 0.9737                       & 0.9709    & 0.9735 & 0.9722 \\ \hline
        lstm\_15 & 0.95                         & 0.9424    & 0.9523 & 0.9473 \\ \hline
        lstm\_16 & 0.9387                       & 0.9340    & 0.9365 & 0.9352 \\ \hline
        cnn\_1   & 0.8725                       & 0.8770    & 0.8492 & 0.8629 \\ \hline
        cnn\_2   & 0.8737                       & 0.8597    & 0.8756 & 0.8676 \\ \hline
        cnn\_3   & 0.8637                       & 0.8493    & 0.8650 & 0.8571 \\ \hline
        cnn\_4   & 0.8937                       & 0.8948    & 0.8783 & 0.8865 \\ \hline
        cnn\_5   & 0.9275                       & 0.9301    & 0.9153 & 0.9226 \\ \hline
        cnn\_6   & 0.965                        & 0.9704    & 0.9550 & 0.9626 \\ \hline
        cnn\_7   & 0.9862                       & 0.9893    & 0.9814 & 0.9853 \\ \hline
        cnn\_8   & 0.8875                       & 0.8730    & 0.8915 & 0.8821 \\ \hline
        cnn\_9   & 0.8825                       & 0.8776    & 0.8730 & 0.8753 \\ \hline
        cnn\_10  & 0.8737                       & 0.8488    & 0.8915 & 0.8696 \\ \hline
        cnn\_11  & 0.8887                       & 0.8714    & 0.8968 & 0.8839 \\ \hline
    \end{tabular}
\end{table}


\begin{table}[p]  \centering
    \caption{Embedding FastText o długości 300, zbiór B, ograniczony do \label{tab:wyniki_fasttext_B_lim}
        równolicznych klas po 2000 rekordów}
    \begin{tabular} {|c|c|c|c|c|c|c|c|c| }    \hline
                 & \multicolumn{4}{c|}{metryki}                               \\ \hline
        sieć     & accuracy                     & precision & recall & f1     \\ \hline
        dense\_1 & 0.9062                       & 0.8796    & 0.9285 & 0.9034 \\ \hline
        dense\_2 & 0.8975                       & 0.8609    & 0.9338 & 0.8959 \\ \hline
        dense\_3 & 0.8975                       & 0.8609    & 0.9338 & 0.8959 \\ \hline
        dense\_4 & 0.9162                       & 0.8877    & 0.9417 & 0.9139 \\ \hline
        lstm\_1  & 0.8737                       & 0.8524    & 0.8862 & 0.8690 \\ \hline
        lstm\_2  & 0.9687                       & 0.9916    & 0.9417 & 0.9660 \\ \hline
        lstm\_3  & 0.965                        & 0.9781    & 0.9470 & 0.9623 \\ \hline
        lstm\_4  & 0.9737                       & 0.9890    & 0.9550 & 0.9717 \\ \hline
        lstm\_5  & 0.9637                       & 0.9754    & 0.9470 & 0.9610 \\ \hline
        lstm\_6  & 0.9475                       & 0.9468    & 0.9417 & 0.9442 \\ \hline
        lstm\_7  & 0.9625                       & 0.9702    & 0.9497 & 0.9598 \\ \hline
        lstm\_8  & 0.9775                       & 0.9838    & 0.9682 & 0.976  \\ \hline
        lstm\_9  & 0.9812                       & 0.9865    & 0.9735 & 0.9800 \\ \hline
        lstm\_10 & 0.9862                       & 0.9972    & 0.9735 & 0.9852 \\ \hline
        lstm\_11 & 0.9787                       & 0.9865    & 0.9682 & 0.9773 \\ \hline
        lstm\_12 & 0.9712                       & 0.9889    & 0.9497 & 0.9689 \\ \hline
        lstm\_13 & 0.9887                       & 0.9973    & 0.9788 & 0.9879 \\ \hline
        lstm\_14 & 0.9862                       & 0.9867    & 0.9841 & 0.9854 \\ \hline
        lstm\_15 & 0.9862                       & 0.9946    & 0.9761 & 0.9853 \\ \hline
        lstm\_16 & 0.98                         & 0.9813    & 0.9761 & 0.9787 \\ \hline
        cnn\_1   & 0.9612                       & 0.9676    & 0.9497 & 0.9586 \\ \hline
        cnn\_2   & 0.9612                       & 0.9651    & 0.9523 & 0.9587 \\ \hline
        cnn\_3   & 0.9437                       & 0.9393    & 0.9417 & 0.9405 \\ \hline
        cnn\_4   & 0.9425                       & 0.9300    & 0.9497 & 0.9397 \\ \hline
        cnn\_5   & 0.9537                       & 0.9595    & 0.9417 & 0.9506 \\ \hline
        cnn\_6   & 0.99                         & 0.9973    & 0.9814 & 0.9893 \\ \hline
        cnn\_7   & 0.99                         & 0.9973    & 0.9814 & 0.9893 \\ \hline
        cnn\_8   & 0.9575                       & 0.9623    & 0.9470 & 0.9546 \\ \hline
        cnn\_9   & 0.9462                       & 0.9419    & 0.9444 & 0.9431 \\ \hline
        cnn\_10  & 0.9437                       & 0.944     & 0.9365 & 0.9402 \\ \hline
        cnn\_11  & 0.9587                       & 0.9700    & 0.9417 & 0.9557 \\ \hline
    \end{tabular}
\end{table}


\begin{table}[p] \label{tab:wyniki_elmo_B_lim} \centering
    \caption{Embedding ELMo o długości 1024, zbiór B, ograniczony do równolicznych klas po 2000 rekordów}
    \begin{tabular} {|c|c|c|c|c|c|c|c|c| }    \hline
                 & \multicolumn{4}{c|}{metryki}                               \\ \hline
        sieć     & accuracy                     & precision & recall & f1     \\ \hline
        dense\_1 & 0.9212                       & 0.9133    & 0.9206 & 0.9169 \\ \hline
        dense\_2 & 0.9225                       & 0.9136    & 0.9232 & 0.9184 \\ \hline
        dense\_3 & 0.8912                       & 0.8759    & 0.8968 & 0.8862 \\ \hline
        dense\_4 & 0.9112                       & 0.9050    & 0.9074 & 0.9062 \\ \hline
        lstm\_1  & 0.9812                       & 0.9814    & 0.9788 & 0.9801 \\ \hline
        lstm\_2  & 0.9812                       & 0.9865    & 0.9735 & 0.9800 \\ \hline
        lstm\_3  & 0.9737                       & 0.9685    & 0.9761 & 0.9723 \\ \hline
        lstm\_4  & 0.955                        & 0.9476    & 0.9576 & 0.9526 \\ \hline
        lstm\_5  & 0.9562                       & 0.9454    & 0.9629 & 0.9541 \\ \hline
        lstm\_6  & 0.975                        & 0.9637    & 0.9841 & 0.9738 \\ \hline
        lstm\_7  & 0.97                         & 0.9682    & 0.9682 & 0.9682 \\ \hline
        lstm\_8  & 0.9862                       & 0.9841    & 0.9867 & 0.9854 \\ \hline
        lstm\_9  & 0.9887                       & 0.9946    & 0.9814 & 0.9880 \\ \hline
        lstm\_10 & 0.9862                       & 0.9919    & 0.9788 & 0.9853 \\ \hline
        lstm\_11 & 0.9912                       & 0.9973    & 0.9841 & 0.9906 \\ \hline
        lstm\_12 & 0.985                        & 0.9972    & 0.9708 & 0.9839 \\ \hline
        lstm\_13 & 0.9837                       & 0.9866    & 0.9788 & 0.9827 \\ \hline
        lstm\_14 & 0.9875                       & 0.9919    & 0.9814 & 0.9867 \\ \hline
        lstm\_15 & 0.9687                       & 0.9632    & 0.9708 & 0.9670 \\ \hline
        lstm\_16 & 0.9837                       & 0.9790    & 0.9867 & 0.9828 \\ \hline
        cnn\_1   & 0.96                         & 0.9726    & 0.9417 & 0.9569 \\ \hline
        cnn\_2   & 0.9675                       & 0.9680    & 0.9629 & 0.9655 \\ \hline
        cnn\_3   & 0.9287                       & 0.9190    & 0.9312 & 0.9250 \\ \hline
        cnn\_4   & 0.9187                       & 0.9195    & 0.9074 & 0.9134 \\ \hline
        cnn\_5   & 0.9425                       & 0.9368    & 0.9417 & 0.9393 \\ \hline
        cnn\_6   & 0.9925                       & 1         & 0.9841 & 0.992  \\ \hline
        cnn\_7   & 0.99                         & 1         & 0.9788 & 0.9893 \\ \hline
        cnn\_8   & 0.9437                       & 0.9370    & 0.9444 & 0.9407 \\ \hline
        cnn\_9   & 0.9512                       & 0.9402    & 0.9576 & 0.9488 \\ \hline
        cnn\_10  & 0.9362                       & 0.9430    & 0.9206 & 0.9317 \\ \hline
        cnn\_11  & 0.985                        & 0.9945    & 0.9735 & 0.9839 \\ \hline
    \end{tabular}
\end{table}









\newpage % Rozdziały zaczynamy od nowej strony.
\section{Szczegółowy opis wykorzystanych architektur sieci}

W tym rozdziale przedstawiona jest dokładna lista wykorzystanych w pracy architektur sieci, ich lista znajduje się w tabeli \ref{table:architektury_sieci}. Nastomiast w tableli \ref{tab:oznaczenia_warstw} został umieszczony opis skrótów zastosowanych do opisu poszczególnych wartw.

% WAŻNE: TODO / FYI The \label should always be after \caption:

%todo: dodać zastąpienie dla XXX

\begin{table}[!h]  \centering
    \caption{Opis oznaczeń wykorzsytanych w tabeli \ref{table:architektury_sieci}}
    \label{tab:oznaczenia_warstw}
    \begin{tabular} {| l | l |} \hline
        Oznaczenie    & Opis oznaczeń                                                          \\ \hline
        Conv(n,k)     & wartwa konwolucyjna, gdzie                                             \\ & \textbf{n} - liczba zasosowanych filtrów, \\ & \textbf{k} - szerokość zastosowanego filtru \\ \hline
        Flatten       & warstwa powodująca spłaszczenie sieci do postaci jednowymiarowej       \\ \hline
        MaxPooling(f) & wartwa max poolingu o oknie wielkości \textbf{f}                                \\ & oraz skoku przesunięcia także wynoszacego \textbf{f}   \\ \hline
        Dense(n)      & wartwa gęsta, o liczbie nueronów wynoszącej \textbf{n}                 \\ \hline
        LSTM(n)       & wartwa LSTM, liczbie nueronów wynoszącej \textbf{n}                            \\ \hline
        Bi-LSTM(n)    & wartwa Bi-LSTM, w której każda warstwa LSTM posiada \textbf{n} neuronów         \\ \hline
        Dropout(d)    & parametr określający jak duża część połączeń zostanie  usunięta między \\ & kolejnymi wartwami, wartość paramtru \textbf{d} określa jaka część losowych \\ & połączeń będzie usunięta w trakcie jednego kroku uczenia, \\ & zawiera się w przedziale od 0 do 1                                                             \\ \hline
    \end{tabular}
\end{table}

\begin{longtable}{| c | m{0.58\linewidth} | r | m{0.1\linewidth} |}
    \caption{Architektury sieci wykorzystane w ramach pracy}
    \label{table:architektury_sieci}        \\
    \hline
    Nazwa sieci & \multicolumn{1}{c|}{Sieć} \\ \hline\hline \endfirsthead

    \endfoot
    \hline \endlastfoot

    $cnn\_1$    & Conv(50,3)                \\
                & Flatten                   \\
                & Dense(50)                 \\
                & Dense(1)                  \\ \hline
    $cnn\_2$    & Conv(300,3)               \\
                & Flatten                   \\
                & Dense(300)                \\
                & Dense(1)                  \\ \hline
    $cnn\_3$    & Conv(300,15)              \\
                & Flatten                   \\
                & Dense(300)                \\
                & Dense(1)                  \\ \hline
    $cnn\_4$    & Conv(300,15)              \\
                & MaxPooling(2)             \\
                & Flatten                   \\
                & Dense(300)                \\
                & Dense(1)                  \\ \hline
    $cnn\_5$    & Conv(300,15)              \\
                & MaxPooling(8)             \\
                & Flatten                   \\
                & Dense(300)                \\
                & Dense(1)                  \\ \hline
    $cnn\_6$    & Conv(300,3)               \\
                & MaxPooling(2)             \\
                & Conv(300,3)               \\
                & MaxPooling(2)             \\
                & Flatten                   \\
                & Dense(300)                \\
                & Dense(1)                  \\ \hline

    $cnn\_7$    & Conv(300,3)               \\
                & MaxPooling(2)             \\
                & Conv(300,3)               \\
                & MaxPooling(2)             \\
                & Conv(300,3)               \\
                & MaxPooling(2)             \\
                & Flatten                   \\
                & Dense(300)                \\
                & Dense(1)                  \\ \hline
    $cnn\_8$    & Conv(300,5)               \\
                & Flatten                   \\
                & Dense(300)                \\
                & Dense(1)                  \\ \hline
    $cnn\_9$    & Conv(300,7)               \\
                & Flatten                   \\
                & Dense(300)                \\
                & Dense(1)                  \\ \hline
    $cnn\_10$   & Conv(300,9)               \\
                & Flatten                   \\
                & Dense(300)                \\
                & Dense(1)                  \\ \hline


    $cnn\_11$   & Conv(500,3)               \\
                & Flatten                   \\
                & Dense(500)                \\
                & Dense(1)                  \\ \hline

    %%%%%%%%%%%%%%%%%%%%%%%%%%%%%%%%%%%%%%%%%%%%%%%%%%%%%%%%%%5

    $dense\_1$  & Dense(30)                 \\
                & Flatten                   \\
                & Dense(1)                  \\ \hline

    $dense\_2$  & Dense(30)                 \\
                & Dense(30)                 \\
                & Flatten                   \\
                & Dense(1)                  \\ \hline

    $dense\_3$  & Dense(300)                \\
                & Dense(300)                \\
                & Flatten                   \\
                & Dense(1)                  \\ \hline

    $dense\_4$  & Dense(300)                \\
                & Dropout(0.2)              \\
                & Dense(300)                \\
                & Dropout(0.2)              \\
                & Flatten                   \\
                & Dense(1)                  \\ \hline
    %%%%%%%%%%%%%%%%%%%%%%%%%%%%%%%%%%%%%%%%%%%%%%%%%%%%%%%%%%
    $lstm\_1$   & LSTM(50)                  \\
                & Dense(5)                  \\
                & Dense(1)                  \\ \hline

    $lstm\_2$   & LSTM(50)                  \\
                & Dense(50)                 \\
                & Dense(1)                  \\ \hline

    $lstm\_3$   & LSTM(30)                  \\
                & Dense(30)                 \\
                & Dense(1)                  \\ \hline
    $lstm\_4$   & LSTM(30)                  \\
                & Dense(1)                  \\ \hline
    $lstm\_5$   & LSTM(30)                  \\
                & Dense(1)                  \\ \hline
    $lstm\_6$   & Bi-LSTM(50))              \\
                & Dense(1)                  \\ \hline
    $lstm\_7$   & Bi-LSTM(25))              \\
                & Dense(1)                  \\ \hline

    %%%%%%%%%%%%%%%%%%%%%%%%%%%%%%%%%%%%%%%%%%%%%%%%%%%%%%%%%%

    $lstm\_8$   & LSTM(300)                 \\
                & Dense(300)                 \\
                & Dense(1)                  \\ \hline

    $lstm\_9$   & LSTM(300)                 \\
                & LSTM(300)                 \\
                & Dense(300)                \\
                & Dense(1)                  \\ \hline

    $lstm\_10$  & LSTM(300)                 \\
                & Dropout(0.2)              \\
                & LSTM(300)                 \\
                & Dropout(0.2)              \\
                & Dense(300)                \\
                & Dense(1)                  \\ \hline
    $lstm\_11$  & LSTM(300)                 \\
                & Dropout(0.2)              \\
                & LSTM(300)                 \\
                & Dropout(0.2)              \\
                & LSTM(300)                 \\
                & Dropout(0.2)              \\
                & Dense(300)                \\
                & Dense(1)                  \\ \hline
    $lstm\_12$  & Bi-LSTM(300)              \\
                & Dropout(0.2)              \\
                & Bi-LSTM(300)              \\
                & Dropout(0.2)              \\
                & Dense(300)                \\
                & Dense(1)                  \\ \hline
    $lstm\_13$  & Bi-LSTM(300)              \\
                & Dropout(0.2)              \\
                & Bi-LSTM(300)              \\
                & Dropout(0.2)              \\
                & Dense(300)                \\
                & Dense(1)                  \\ \hline

    $lstm\_14$  & Bi-LSTM(500)              \\
                & Dropout(0.2)              \\
                & Dense(500)                \\
                & Dense(1)                  \\ \hline

    $lstm\_15$  & Bi-LSTM(300)              \\
                & Dropout(0.2)              \\
                & Bi-LSTM(300)              \\
                & Dropout(0.2)              \\
                & Dense(300)                \\
                & Dense(1)                  \\ \hline

    $lstm\_16$  & Bi-LSTM(300)              \\
                & Dropout(0.2)              \\
                & Bi-LSTM(300)              \\
                & Dropout(0.2)              \\
                & Bi-LSTM(300)              \\
                & Dropout(0.2)              \\
                & Dense(300)                \\
                & Dense(1)                  %\\ \hline
\end{longtable}





% \newpage % Rozdziały zaczynamy od nowej strony.
\section{Praefatio}
dfdf

% \lipsum[1] \cite{goossens93}
% \begin{figure}[!h]
%     \label{fig:anzelm}
%     \centering \includegraphics[width=0.5\linewidth]{logopw.png}
%     \caption{Tradycyjne godło Politechniki Warszawskiej.}
% \end{figure}
% \lipsum[2-10]


\begin{table}[!h] \label{tab:tabela1} \centering
    \caption{Przykładowa tabela.}
    \begin{tabular} {| c | c |} \hline
        % Kolumna 1 & Kolumna 2 & Liczba \\ \hline\hline
        0.456 & 0.75 \\ \hline
        0.723 & 0.2  \\ \hline
    \end{tabular}
\end{table}
% \input{tex/92-de-finibus}
% \input{tex/93-code-listings}

\newpage % Rozdziały zaczynamy od nowej strony
\section{Podsumowanie}          % Można też pisać rozdziały w jednym pliku.
% \lipsum[5-10]
% PODSUMOWANIE

W ramach pracy została dokonana kompleksowa analiza pozwalająca na zbudowanie modelu sieci neuronowej umożliwiającej rozwiązanie zadania klasyfikacji tekstów ironicznych. Zostały w niej zaprezentowane kolejne kroki, które należy wykonać by pozwolić sieci na interpretacje zdania. W pierwszej kolejności zostały omówione czynności niezbędne w ramach obróbki wstępnej danych. Następnie, przedstawiono sposoby przeniesienia zmiennych kategorycznych w postaci słów do przestrzeni wektorów liczbowych przy zachowaniu jak największej liczby informacji na temat podobieństwa poszczególnych słów. W końcowym etapie porównano kilka możliwych architektur sieci i ich skuteczność poprzez wykorzystanie różnych popularnych metryk. Ponadto zbadano wpływ cech morfosyntaktycznych na jakość klasyfikacji w problemie detekcji ironii. W oparciu o drobiazgowe wyniki nie dostrzeżono znaczącej poprawy jakości klasyfikacji dla żadnego z wykorzystanych zbiorów. 

Niestety w ramach pracy nie wszystko poszło zgodnie z założeniami początkowymi. Nie udało się na przykład dokonać detekcji ironii w ramach języka polskiego. Było to spowodowane w dużej mierze brakiem łatwo dostępnych, bezpłatnych danych mogących posłużyć do analizy. W przypadku rozwiązania tego problemu, otwiera się bardzo ciekawy kierunek dalszego rozwoju tej pracy. 

Innym elementem mogącym być dalszym kierunkiem rozwoju pracy jest opracowanie modelu pozwalającego na większą generalizację klasyfikacji dokonywanej przez model. W obecnej konfiguracji, sieć nie rodzi sobie dobrze z klasyfikacją wypowiedzi pochodzących z poza korpusu wykorzystanego w procesie treningu i testowania. Najprawdopodobniej wynika to ze zbyt małej różnorodności wypowiedzi w ramach zbioru danych. W celu rozwiązania tego problemu należałoby stworzyć korpus o bardziej różnorodnej budowie. 
% //TODO

%--------------------------------------------
% Literatura
%--------------------------------------------
\cleardoublepage % Zaczynamy od nieparzystej strony
\printbibliography

%--------------------------------------------
% Spisy (opcjonalne)
%--------------------------------------------
\newpage
\pagestyle{plain}

% Wykaz symboli i skrótów.
% Pamiętaj, żeby posortować symbole alfabetycznie
% we własnym zakresie. Ponieważ mało kto używa takiego wykazu,
% uznałem, że robienie automatycznie sortowanej listy
% na poziomie LaTeXa to za duży overkill.
% Makro \acronymlist generuje właściwy tytuł sekcji,
% w zależności od języka.
% Makro \acronym dodaje skrót/symbol do listy,
% zapewniając podstawowe formatowanie.
% //AB
% \vspace{0.8cm}
% \acronymlist
% \acronym{EiTI}{Wydział Elektroniki i Technik Informacyjnych}
% \acronym{PW}{Politechnika Warszawska}
% \acronym{WEIRD}{ang. \emph{Western, Educated, Industrialized, Rich and Democratic}}

\listoffigures              % Spis obrazków.
\vspace{1cm}                % vertical space
\listoftables               % Spis tabel.
% \vspace{1cm}                % vertical space
% \listofappendices           % Spis załączników

% Załączniki
% \newpage
% \appendix{Nazwa załącznika 1}
% \lipsum[1-8]

% \newpage
% \appendix{Nazwa załącznika 2}
% \lipsum[1-4]

% Używając powyższych spisów jako szablonu,
% możesz tu dodać swój własny wykaz bądź listę,
% np. spis algorytmów.

\end{document} % Dobranoc.
